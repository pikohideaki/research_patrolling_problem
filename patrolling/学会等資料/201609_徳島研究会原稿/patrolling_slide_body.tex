
\section{Introduction}




\begin{frame}
\ruby{警邏}{けい|ら}(patrolling)とは
\begin{itemize}
	\item 1人または複数の巡査により
	\item 領域内のあらゆる場所を十分な頻度で訪問
\end{itemize}
警邏する対象
\begin{itemize}
	\item グラフの頂点
	\item 線分や閉路などの全体
	\item 二次元の領域
\end{itemize}
% 図など追加
今回は,グラフの頂点の警邏を考える.
\end{frame}




\begin{frame}{問題設定}
\begin{itemize}
	\item 入力
	\begin{itemize}
		\item グラフ
		\item 各頂点の{\alert {\timelimit} }
		% \begin{itemize}
		% 	\item 点 $v_i$ を警備するためには
		% 	連続した2回の訪問の間隔を{\timelimit} $q_i$ 以内にしなければならない
		% \end{itemize}
		\item 各頂点の利得
		\item 巡査の数
	\end{itemize}
	\item 目的
	\begin{itemize}
		\item グラフの全頂点を警邏できるかどうか判定(\decisionpp)
		\item 警邏できる頂点部分集合で利得の和が最大のものを求める(\optpp)
		\item \optpp は \decisionpp の一般化
		(\optpp の出力が全頂点の利得の和と等しいか比べれば
		\decisionpp の解となる)
	\end{itemize}
\end{itemize}
\end{frame}




\begin{frame}{既存の結果}
\begin{itemize}
	\item 図形(図使用)
	\item 簡単な場合 ({\timelimit}が全て等しい場合など)
\end{itemize}
\end{frame}




\begin{frame}{既存の結果}
% ・):巡査が1人の場合はP,複数の場合は未解決 \\
% ・:巡査が1人で各頂点の{\timelimit}・利得が全て等しい場合のみP, 
% それ以外の巡査が複数である場合や利得・{\timelimit}のいずれかが一般の場合はいずれもNP困難 \\
% 一般のグラフ:巡査が1人で各頂点の{\timelimit}・利得が全て等しい場合でもNP困難

\begin{table}
	\begin{tabular}{|c|c|c|c|c|}
	\hline       & \multicolumn{2}{|c|}{{\timelimit}が全て等しい場合}
	             & \multicolumn{2}{|c|}{{\timelimit}が一般の場合} \\
	\hline       & 巡査1人      & 巡査複数 & 巡査1人 & 巡査複数 \\
	\hline 線分  & P            & 未解決   & P       & 未解決   \\
	\hline 閉路  & P            & 未解決   & P       & 未解決   \\
	\hline 星    & NP困難(※) & NP困難   & NP困難  & NP困難   \\
	\hline 木    & NP困難(※) & NP困難   & NP困難  & NP困難   \\
	\hline 一般  & NP困難       & NP困難   & NP困難  & NP困難   \\
	\hline
	\end{tabular}
\end{table}

Coene : "Charlemagne's challenge: the periodic latency problem".
\end{frame}




\begin{frame}{既存の結果+本研究の結果}
\begin{itemize}
	\item Lineの複数のところ
	\item UStar
\end{itemize}
\end{frame}






\section{Line}


\subsection{{\timelimit}がすべて等しい場合}


\begin{frame}{}
	
\end{frame}




\subsection{{\timelimit}が一般の場合}


\begin{frame}{}
	
\end{frame}




\subsubsection{最初の訪問時刻指定,周期ちょうど毎に訪問}


\begin{frame}{}
	
\end{frame}






\section{UStar}



\begin{frame}{}
	
\end{frame}





\subsection{{\timelimit}がすべて等しい場合}



\begin{frame}{}
	
\end{frame}




\subsection{{\timelimit}が一般の場合}


\begin{frame}{}
	
\end{frame}




\subsubsection{最初の訪問時刻指定,周期ちょうど毎に訪問}


\begin{frame}{}
	
\end{frame}




\subsubsection{周期ちょうど毎に訪問}


\begin{frame}{}
	
\end{frame}




