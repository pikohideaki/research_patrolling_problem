\begin{abstract}
1人または複数の巡査が所与の領域を動き回り,その領域内の指定された場所を十分な頻度で訪問することで,これを守備,監督することを警邏という.

% 警邏に関連する様々な問題が考えられており,

Coeneらは次のような警邏問題を考えた.

\begin{patrollingproblem}
巡査の人数$m$と
辺に非負整数の長さがついた無向グラフ$G$(各点の利得と許容訪問間隔を含む)
が与えられる.
$G$の点集合$V$の部分集合$W$であって, $m$人の巡査により警邏可能
(すなわち,$W$の任意の点について,
その点の許容訪問間隔以上の時間過去に訪問されていないような時刻が存在しない)
であるもののうち,利得の和が最大となるものを求めよ.
どの巡査も$G$上を速さ$1$以下で動くとする.
\end{patrollingproblem}

%
% この問題は入力が一般のグラフの場合はNP困難であることが示されているため,
% グラフが線分の場合など,距離に制約を加えた場合の形状が問題となる.
%
Coeneらはこの警邏問題に対し,
$V$の任意の点$v$について,$v$を警備する巡査は高々1人でなければならない(非協力)という仮定をさらに加えており,
これによりいくつかのグラフについて多項式時間アルゴリズムやNP困難性を示している.


本研究ではこの非協力警邏問題の発展として,
非協力の制約を無くした場合(
2人以上の巡査が協力して警備する点があってもよい)の警邏問題を考える.
非協力の場合と同様に
一般のグラフではNP困難性が示されているため,
形状や巡査の数などについて様々な制約を加えた場合を考える.
NP困難性や多項式時間アルゴリズムを示すのが難しかった場合については,
許容訪問間隔により定義している各点の警備の条件を改変した問題も考えながら
% 許容訪問間隔の代わりに
% ちょうど訪問しなければならない時刻の列が与えられる
計算量クラスの評価を試みる.

今後の課題として,
上記のうち計算量クラスの評価を与えられていない場合を解決することを目標としながら,
まずは先ほどの警備の条件を変えた問題について調べていく予定である.



% 今後の課題
% * 円周の警邏(塀の警邏などから)
% * 巡査の動く速さが異なる場合?
% * Optimal_Patrolling_of_Fragmented_Boundaries
%   * 警邏問題は art gallery question の一種
%   * 複数の巡査の最高速度がすべて1ではなく異なる場合など



% Coeneらは,
% 辺の長さの与えられた無向グラフと各頂点の持つ利得・許容訪問間隔,巡査の人数が与えられたとき,
% 巡査がグラフ上を速さ1以下で動き頂点を訪問することで
% 警備できる頂点集合のうち,利得の合計を最大化するものを求める警邏問題を考えた.
% ある頂点を警備できているとは,
% どの連続した2回の訪問も間隔がその点の許容訪問間隔以内となるように
% 訪問し続けられることと定義される.
% グラフが
% Line(1つの線分上にすべての頂点があるグラフ)または
% Circle(Line の両端をつなぐ辺を足したグラフ)
% で巡査が1人の場合には多項式時間アルゴリズム,
% 星,木,一般の場合についてはNP困難性が示されている.
% 本稿ではこの問題を取り上げ,
% 多項式時間で解けるか否か未解決であった
% Line で巡査が複数の場合や,
% 既にNP困難性が示されている星よりも単純な,
% 星で枝の長さがすべて等しい図形(本稿では UStar と呼ぶことにする)について
% 詳しく調べることにする.
% そのままの問題設定では解決できなかった部分については,
% 許容訪問間隔の代わりに厳密訪問間隔が与えられたときに,
% 最初の訪問時刻からその厳密訪問間隔ごとの時刻は必ず訪問しなければならないという問題,
% さらに最初の訪問時刻も与えられる問題も考える.
\end{abstract}
