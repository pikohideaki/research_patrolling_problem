
\section{Star}
グラフの形状がStarの場合については,
利得か{\timelimit}のいずれかが一般であれば,
{\patrolling}は巡査が1人であってもNP困難であることがわかっている\cite{coene2011charlemagne}.
そこで,ここでは巡査数が一般であって,
利得と{\timelimit}がすべて等しい場合を考える.
%
この場合,非協力警邏問題ではNP困難になることが
Coeneら\cite{coene2011charlemagne}により示されているが,
今回考えている協力警邏問題では多項式時間で解くことができる.
%
これは,
非協力の場合にはうまく分担する方法を見つける必要があるのに対し,
協力警邏の場合には単純で最適な警邏を構成できることが理由となっている.

以下では
% そのような単純な警邏の仕方を示していくが,
{\timelimit}を$Q$とするとき,
「複数人の巡査が間隔$Q$ずつ離れて列を成し巡回する」動き
と「点に1人の巡査が常駐する」という2種類の運行の組合せで
最適な運行が得られることを示す.

% 隣接枝が一定以上長い点は
% 枝の往復にかかる時間が大きくなるため,
% 巡回に含めずに巡査を1人常駐させる方が
% この点の1回あたりの警備にかかる時間を減らすことができる.



\begin{theo}
    \label{theo:StarEqualProfitTimelimit}
    グラフの形状がStarで利得と{\timelimit}がすべて等しい場合,
    {\patrolling}は多項式時間で解くことができる.
\end{theo}


\begin{proof}[証明]

    巡査数を$m$, 全頂点の{\timelimit}を$Q$, 
    頂点$v_i (i \in \mathset{1,\ldots, n})$に隣接する枝を$e_i$, その長さを$d_i$とする.
    枝の長さは$d_1 \leq d_2 \leq \cdots \leq d_n$であるとする.

    まず,
    すべての頂点の利得と{\timelimit}が等しいので,枝の短い頂点から選べばよいことが分かる.
    実際,ある警邏において警備している頂点$v_i$と警備していない頂点$v_j$であって
    隣接枝$e_i$, $e_j$の長さが$d_i > d_j$となっているようなものがあったとき,
    $v_j$を訪問していた時刻に代わりに$v_i$を訪問するようにすべての巡査の動きを変えることができる.




    まず,$d_i > Q/2$であるような頂点$v_i$は,警備するならば巡査が1人常駐するとしてよい.
    これは次のように示される.
    %
    (i)ある運行において$v_i$が1人の巡査$s$により単独警備されるとすると,
    $v_i$を訪問してから別の頂点$v_j$を訪問して再び$v_i$に戻ってくるには$2d_i$以上の時間がかかり
    $2d_i > Q$より$v_i$が警備できなくなってしまうため$v_i$以外の頂点を警備することができないので
    $v_i$のみを警備すればよく,これは$v_i$に常駐すれば十分である.
    %
    (ii)もし$v_i$が2人以上の巡査により警備されるとすると,
    ある巡査$s_a$が時刻$t_a$に$v_i$を訪問してから時間$Q$以内の時刻$t_b$に
    別の巡査$s_b$が$v_i$を訪問するという状況が発生するが,
    $s_a$が枝
    $s_b$は端点$v_i$を含む枝$e_i$上のある点に同時に存在するような時刻が
    $s$が
    $s$は$s'$とすれ違うことなく$v_i$以外の頂点を訪問



    2人の巡査がすれ違う動きは互いに動きを交換して引き返す動きに変換しているとすると
    $s$と$s'$が枝$e_i$上で1度以上すれ違わない限り$s'$が$v_i$を訪問するときには$s$も
    $v_i$を訪問しているので,
    $v_i$は$s$のみにより警備できており,$s$も$v_i$以外を警備していない動きとなる.



    よって,あとは隣接している枝の短い頂点から何個の頂点を選べるかを計算できればよい.
    %
    はじめに,頂点を枝の長さの昇順でソートし,枝の短いものから順に$v_1,v_2, \ldots, v_n$とする.
    これらを枝の長さが$Q/2$以下のグループ$V_1 = \mathset{v_1, \ldots, v_k}$と
    それ以外のグループ$V_2 = \mathset{ v_{k + 1}, \ldots, v_n }$に分ける.
    $V_2$の頂点は,$V_1$の全頂点を$m - 1$人以下の巡査により警備できる場合のみ,
    残りの巡査の人数分$V_2$の頂点を選び1人ずつ巡査を常駐させることで警備すればよいので,
    まず$V_1$のすべての頂点を警備できる最小の巡査数$m'$を求める必要がある.

    $m' \leq m$であれば利得(警備できる頂点数)は$k + (m - m')$となる.
    $m' > m$であれば$V_1$のうち

\end{proof}

