\section{{\graphStar}}
\label{section: star}

グラフの形状が{\graphStar}の場合については,
利得か{\idletime}のいずれかが一般であれば,
{\patProb}は巡査が1人であってもNP困難であることが分かっている~\cite{coene2011charlemagne}.
よって,本稿の{\patProb}については,巡査数が一般であって,
全点の利得と{\idletime}が等しい場合を調べる.

{\independentPatProb}においては,グラフが{\graphStar}で巡査数が一般の場合は
利得と{\idletime}がすべて等しくてもNP困難になることが
Coeneら~\cite{coene2011charlemagne}により示されているが,
一方で同じ条件における{\patProb}の場合は次が成り立つ.

\begin{theo}
  \label{theo:StarEqualProfitTimelimit}
  グラフの形状が{\graphStar}で全点の利得と{\idletime}が等しい場合,
  {\patProb}は(巡査数が一般であっても)多項式時間で解くことができる.
\end{theo}


{\graphLine}の場合では協力の発生によって複雑な運行による警邏が発生した状況から考えると,
{\independentPatProb}より{\patProb}の方が簡単に解けるというのは意外な結果に思われるが,
{\graphStar}の場合は,{\independentPatProb}では
うまく頂点集合を分割しなければならないことを用いて分割問題を帰着することができるためNP困難になるのに対し,
{\patProb}ではある単純な運行が最適となるため簡単に解くことができる.


本節では,頂点$v$に隣接する辺を$e_v$, その長さを$d_v$のように書く.

\begin{lemm}
  \label{lemm:condition_of_guarding_star}
  グラフの形状が{\graphStar}のときの{\patProb}において,
  全点の{\idletime}が$Q$のとき,
  点集合$V$の任意の部分集合$W$について
  $$
    \sum_{v \in W} \min(2d_v, Q) \leq mQ
    \iff \textrm{$m$人の巡査により$W$の全点を警邏できる}
  $$
  が成り立つ.
\end{lemm}





\begin{proof}[証明]
  % ⇒
  $(\Rightarrow)$
  $\sum_{v \in W} \min(2d_v, Q) \leq mQ$が成り立つとき,
  $m$人の巡査の運行を次のように定めれば$W$の全点を警邏できる.
  $W' := \{ v \in W \mid 2d_v \geq Q \}$とする.
  まず,$\card{W'}$人の巡査を$W'$の各点に1人ずつ配備し常駐させることで警備する.
  このとき,$\card{W'}Q = \sum_{v \in W'} Q \leq \sum_{v \in W} \min(2d_v, Q) \leq mQ$
  より$\card{W'} \leq m$である.
  これにより$W'$の全点は$\card{W'}$人の巡査により警備される.
  %
  残りの$2d_v < Q$である頂点$v \in W \setminus W'$は,
  残りの$m - \card{W'}$人の巡査により警邏しなければならないが,
  $\sum_{v \in W} \min(2d_v, Q) \leq mQ$より
  $\sum_{v \in W \setminus W'} 2d_v \leq (m - \card{W'})Q$
  であるから,
  1人の巡査が速さ1で$W \setminus W'$の全点をちょうど1度ずつ訪問するのにかかる時間は
  $(m - \card{W'})Q$以下となるので(中心点と点$v$を一往復するには$2d_v$の時間を要する),
  $m - \card{W'}$人の巡査全員が時間$Q$ずつずらしてこの動きを繰り返すことで
  どの点も時間$Q$以上放置せずに訪問し続けることができ,
  これにより$W \setminus W'$の全点も警備される.

  % ⇐
  $(\Leftarrow)$
  対偶を示す.
  まず,ある点$v$が警備されているとは,
  どの長さ$Q$の時間区間にも少なくとも1度巡査の訪問を受けることであるが,
  点$v$が警備されているとき,どの長さ$Q$の時間にも
  $\min(2d_v, Q)$の時間は(少なくとも一人の)巡査が$e_v$上(端点のうち中心点は含まず$v$のみを含む)に存在する.
  これは後で示す.
  各点$v$に対して定まる時間$Q$あたりの巡査の滞在時間$\min(2d_v, Q)$は
  点$v$の警備に必要なコストと考えることができる.
  %
  $W$の全点の警邏に必要なコストは,各点の警備コストの和$\sum_{v \in W} \min(2d_v, Q)$であり,
  $m$人の巡査の持つ時間(資源)の和は時間$Q$あたり$mQ$であるが,
  各巡査は同時に2つ以上の辺上には存在しえないため(中心点は辺に含まない),
  $\sum_{v \in W} \min(2d_v, Q) > mQ$のとき
  $W$の全点を警邏することはできない.
  %
  
  最後に,
  点$v$が警備されているとき,どの長さ$Q$の時間にも
  $\min(2d_v, Q)$の時間は(少なくとも一人の)巡査が$e_v$上に存在する
  ことを示す.
  %
  (i) $v$が$2d_v \geq Q$を満たすとき,
    もし$v$の隣接辺$e_v$上(端点を含む)に一人も巡査が存在しない時刻$s$があるとすると,
    $v$を訪問した$s$以前で最後の時刻と$s$以後で最初の時刻の間隔は$2d_v \geq Q$より長いため,
    $v$が警備されていることに反する.
    よってこの場合は
    隣接辺$e_v$上には常にいずれかの巡査が存在する必要がある.
  %
  (ii) $v$が$2d_v < Q$を満たすとき,
    長さ$Q$の時間区間$[t, t + Q]$を任意に選ぶ.
    警備の条件から$v$は$[t, t + Q]$に少なくとも1回訪問されるが,
    その時刻によって以下の場合を考える.
    (a) $[t + d_v, t + Q - d_v]$に1回以上訪問されるとき,
      その訪問時刻を任意に1つ選び$s$とすると
      $s$の前後の少なくとも$d_v$ずつの時間は
      巡査は辺$e_v$上に存在し,これは$[t, t + Q]$に含まれる.
    (b) $[t + d_v, t + Q - d_v]$に1度も訪問されないときは,
      $[t, t + d_v)$か$(t + Q - d_v, t + Q]$に少なくとも1回訪問される.
      (b1) $[t, t + d_v)$と$(t + Q - d_v, t + Q]$でそれぞれで少なくとも1回ずつ訪問されるときは
        $[t, t + d_v]$と$[t + Q - d_v, t + Q]$に巡査が$e_v$上に存在するので
        巡査の滞在時間は$2d_v$以上となる.
      (b2) $(t + Q - d_v, t + Q]$に一度も訪問されないとき,
        $[t, t + d_v)$に含まれる最後の訪問時刻を$s$とすると,
        点$v$の警備の条件と場合分けの仮定から$s$の次の訪問時刻$u$は
        $t + Q < u \leq s + Q$を満たす.
        $s$と$u$それぞれの前後$d_v$の時間$[s - d_v, s + d_v], [u - d_v, u + d_v]$には
        巡査が辺$e_v$に存在するが,
        このうち$[t, t + Q]$に含まれるのは$[t, s + d_v], [u - d_v, t + Q]$であり,
        その時間の和は
        $(s + d_v - t) + (t + Q - (u - d_v)) = 2d_v + (Q - (u - s)) \geq 2d_v$
        より$2d_v$以上となる.
      (b3) $[t, t + d_v)$に1度も訪問されないとき,
        $(t + Q - d_v, t + Q]$に含まれる最初の訪問時刻とその1つ前の訪問時刻を考えると
        (b2)と同様に巡査の滞在時間の和は$2d_v$以上となる.

  % comment : 非存在時間の上限から示す方が楽かと考えたが,
  %           訪問はしないが辺上に一瞬だけ存在するような巡査を省く必要があるので面倒
  %
  % $[t + 2d_v, t + Q - 2d_v]$は$e_v$上に巡査が一人も存在しない時間となるが,
  % $[t, t + Q]$に含まれる巡査非存在時間が$Q - 2d_v$より長いとすると,これは$[t, t + Q]$においては連続しており
\end{proof}



補題\ref{lemm:condition_of_guarding_star}より
{\graphStar}の任意の点部分集合$W$が警邏可能であるかを
$W$の点の隣接辺の長さだけから簡単に計算できることが分かった.
定理\ref{theo:StarEqualProfitTimelimit}では,
全点の利得と{\idletime}が等しい場合を考えているので
警邏する部分集合としては隣接辺の短い点から順に選べばよく
(隣接辺のより長い点$v_1$とより短い点$v_2$があるとき,
$v_1$を警備して$v_2$を警備しない運行は常に$v_1$を警備する代わりに$v_2$を警備する運行に変換できる),
警邏できる最大の部分集合は$n$を点の数として$O(n \log n)$で計算できる
(もちろん,ヒープソートなどを用いれば計算量はさらに改善される).
以上から定理\ref{theo:StarEqualProfitTimelimit}が示された.
