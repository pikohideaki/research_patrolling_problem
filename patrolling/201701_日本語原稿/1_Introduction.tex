
\section{はじめに}
% 警邏とは
所与の領域を1人または複数の巡査が動き回り,
その領域内の指定された場所を十分な頻度で訪れることを
警邏(patrolling)という\cite{chen2013fence, coene2011charlemagne, czyzowicz2011boundary}.


本稿では,与えられた距離空間$U$内を速さ$1$以下の巡査$m$人が動きまわることにより,
集合$V \subseteq U$に属する多くの点に十分な頻度で訪れるという目標を考える.
すなわち次のような問題である.

巡査$i$の$U$上の運行$a_i: \Rset \to U$とは,
各時刻$t \in \Rset$における位置$a _i (t) \in U$を定めるものであって,
任意の時刻$s$,$t \in \Rset$に対し$a _i (s)$と$a _i (t)$の距離が$\lvert s - t \rvert$を超えないものをいう.
%
巡査$m$人による$U$上の運行とは,
各巡査$i \in \{1, \ldots, m\}$の運行の組
$(a _i) _{i \in \{1, \ldots, m\}}$
をいう.
%
% まず,巡査$m$人による$U$上の運行とは,
% 各巡査$i \in \{1, \ldots, m\}$の
% 各時刻$t \in \Rset$における位置$a _i (t) \in U$を定めたものであって,
% 任意の時刻$s$,$t \in \Rset$に対し$a _i (s)$と$a _i (t)$の距離が$\lvert s - t \rvert$を超えないものをいう.
%
運行$A$
% $A = (a _i) _{i \in \{1, \ldots, m\}}$
が点$v \in U$を間隔$q \geq 0$で警備するとは,
長さ$q$のどの時間区間にも
いずれかの巡査が$v$を少くとも一度は訪れる
(任意の時刻$t \in \Rset$に対して
巡査$i$と時刻$\tau \in [t, t + q)$が存在し
$a _i (\tau) = v$)ことをいう.

$U$の有限な部分集合$V$があり,$V$の各点には利得および{\timelimit}と呼ばれる非負整数が定まっている.
運行$A$が点集合$W \subseteq V$を警邏するとは,各点$v \in W$に対し,
$A$が$v$をその{\timelimit}以下の間隔で警備することをいう.
そのような運行が存在するとき$W$は$m$人により警邏可能であるという.

\begin{cooperativepatrollingproblem}
    巡査の人数$m$と距離空間$U$内の点集合$V$(各点の利得と{\timelimit}を含む)が
    与えられたとき,
    警邏可能な頂点集合のうち利得の和が最大となるものを求めよ.
\end{cooperativepatrollingproblem}

距離空間$U$といっても,$V$の点どうしの距離のみが重要である.
そこでこの問題の入力は,
$V$を頂点集合とし辺に非負整数の長さがついた無向グラフと考えることにする.



%%%%%%%%%%%%%%%%%%%%%%%%%%%%%%%%%%%%%%%%%%%%%%%%%%%%%
%%% old
%%% \section{はじめに}
%%% % 警邏とは
%%% 所与の領域を1人または複数の巡査が動き回り,
%%% その領域内の指定された場所を十分な頻度で訪問することを
%%% 警邏(patrolling)という\cite{chen2013fence, coene2011charlemagne, czyzowicz2011boundary}.
%%%
%%%
%%% \red{「距離空間$M$の有限部分集合$V$が与えられる」などにする?}
%%% 本稿では,辺に非負の長さがついた無向グラフ$G$上を,
%%% 速さ$1$の巡査が何人かで動きまわることで,
%%% 多くの頂点が十分な頻度で訪問されるようにする問題を考える.
%%% 巡査$m$人による$G$上の運行とは,
%%% 各巡査$i \in \{1, \ldots, m\}$の
%%% 各時刻$t \in \Rset$における位置$a _i (t)$を定めたものをいう.
%%% 位置$a _i (t)$は$G$の頂点または辺上の一点であり,
%%% 任意の時刻$s$,$t$に対し,
%%% $a _i (s)$と$a _i (t)$の間の$G$上での距離は$\abs{s - t}$を超えない.
%%%
%%%
%%% $G$の各頂点には利得および{\timelimit}と呼ばれる非負整数が与えられている.
%%% ある運行によって頂点$v$が警備されるとは,
%%% $v$の{\timelimit}を$q$とするとき,
%%% 長さ$q$のどの時間区間を見ても,
%%% いずれかの巡査が$v$を少くとも一度は訪問する
%%% (すなわち任意の時刻$t \in \Rset$に対して
%%% 巡査$i$と時刻$\tau \in [t, t + q)$が存在し
%%% $a _i (\tau) = v$)ことをいう.
%%% 運行$A$が頂点の集合$W$を警邏するとは,$A$が$W$の全頂点を警備することをいう.
%%% $W$が巡査$m$人により警邏可能であるとは,
%%% $W$を警邏する巡査$m$人による運行が存在することをいう.
%%% \red{「警備」の多重定義}
%%%
%%% 本稿では次の問題を考える.
%%%
%%% \begin{cooperativepatrollingproblem}
%%%     巡査の人数$m$とグラフ$G$(辺の長さ,頂点の利得と{\timelimit}を含む)が
%%%     与えられたとき,
%%%     警邏可能な頂点集合のうち利得の和が最大となるものを求めよ.
%%% \end{cooperativepatrollingproblem}
%%%%%%%%%%%%%%%%%%%%%%%%%%%%%%%%%%%%%%%%%%%%%%%%%%%%%%%


この問題は,巡査が一人かつ
全頂点の利得と{\timelimit}が等しい場合に限っても,
ハミルトン路問題からの帰着により
NP困難である\cite[Theorem~8]{coene2011charlemagne}.
そこでグラフの形状を限ったときにどのようになるかを調べる.

一つの頂点が複数の巡査の訪問により警備され得ることに注意されたい.
例えば図\ref{figure: }では,…….
\red{(←適当な例を図に描く。)}

Coeneら\cite{coene2011charlemagne}は似た問題を扱っているが,
このような協力を許さず,
各頂点を専ら一人の巡査が「担当」することを要求している.
つまり,各頂点$v \in W$が単独警備されること,すなわち
$v$によって決るある一人の巡査が,
$v$の{\timelimit}以上の長さのどの時間区間においても
$v$を訪問することを要求しているのである.
対比のため本稿ではこの問題を非協力警備問題と呼ぶことにする
(\cite{coene2011charlemagne}ではMPLPPと称している).
Coeneら\cite{coene2011charlemagne}の諸結果においては,
この非協力という限定が,
多項式時間算法の設計にも困難性の証明にも重要な役割を果している.
この限定を外したときの様子を調べるのが本稿の目的である.

%% 複数の巡査の協力を考える本稿の問題においては,
%% 巡査が1人の場合でNP困難性が示されていない
%% 線分・閉路の場合と星・木で全頂点の利得と{\timelimit}が等しい場合
%% が未解決である.

本稿ではグラフ$G$の形状として
線分,星と,枝の長さがすべて等しい完全グラフの3種類を扱うこととし
\red{(←定義が必要ですね。図にしましょうか。)},
以降はそれぞれを Line, Star, {\unit}と呼ぶ.
{\unit}は,その各辺の長さを$d$とすると,
同じ頂点数で辺の長さがすべて$d/2$というStarの特別な場合と考えることができる.

% 頂点の警備にかかるコストにはその頂点の{\timelimit}とその頂点までの移動距離という2種類の制約があるが,
% {\unit}はそのうち頂点までの移動距離が全頂点で等しいという場合となる.

%% 頂点の警邏を考える今回の問題においては,グラフの形状は頂点間の移動距離の性質としてのみ意味を持つ.

% 巡査協力なしの結果との比較

協力警邏問題についての我々の結果と,
非協力警邏問題についてのCoeneらの結果を,
グラフの形ごとに比較すると次のようになる.
それぞれ\ref{section: line},\ref{},\ref{}節で述べる.
\begin{itemize}
\item 
Lineでは,
非協力警邏問題は%巡査数・利得・{\timelimit}がすべて一般の場合であっても
動的計画法により多項式時間で解けることが
示されていた\cite[Theorem~11]{coene2011charlemagne}が,
その正しさは非協力の設定に強く依存している.
本稿では協力警邏問題について,
全頂点の{\timelimit}が等しい場合にはPであることを示す(定理\ref{theo:LineEqualTimelimit}).
\item
Starでは,
全頂点の利得と{\timelimit}がすべて等しい場合に限っても,
非協力警邏問題はNP困難であることが示されていた\cite[Theorem~10]{coene2011charlemagne}.
本稿では,この場合の協力警邏問題はPとなるという興味深い結果を得る(定理**).
なお利得または{\timelimit}を一般にすると,
巡査が一人であっても(したがって協力・非協力に関わらず)
NP困難であることがわかっている\cite[Theorems 5 and 6]{coene2011charlemagne}.
\item 
{\unit}では,
全頂点の{\timelimit}が等しい場合は\patrolling がPであることを示す(定理**).
Starでは全頂点の{\timelimit}が等しくても利得が一般だとNP困難になるので,
これにより{\unit}はStarよりも簡単に解ける場合となっていることが分かる.
\end{itemize}

Lineと{\unit}については
{\timelimit}が一般の場合については多項式時間アルゴリズムやNP困難性を示すのが難しかったため,
{\timelimit}の代わりに「{\interval}」というものを考え,
最初の訪問時刻から{\interval}ごとの時刻ちょうどに訪問し続けること
を警備の条件とする問題も考えた.




\subsection*{関連研究}

\red{(あまり本筋に関係ない関連研究は、論文冒頭ではなくこの辺に書くのも手)}

% 警邏に関する研究には様々な問題設定があり,
% 例えば線分や閉路のような交わりの無い1次元的な領域のすべての点を警邏する
% 塀の警邏(Fence Patrolling)問題~\cite{chen2013fence, czyzowicz2011boundary}や,
% より一般的なグラフで辺全体ではなく頂点を警備する警邏問題~\cite{coene2011charlemagne},
% グラフと巡査が与えられて警邏可能かを判定する問題だけでなく,
% 塀の警邏問題においてなるべく長い塀を警邏する問題~\cite{czyzowicz2011boundary}や
% 全体の訪問の待ち時間の最大値を最小化する問題~\cite{chen2013fence}
% なども考えられている.
% \red{(加筆予定)}

また,LineやStarは木の特別な場合である.
