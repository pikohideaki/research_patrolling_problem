\subsection*{関連研究}

\red{(あまり本筋に関係ない関連研究は、論文冒頭ではなくこの辺に書くのも手)}

% 警邏に関する研究には様々な問題設定があり,
% 例えば線分や閉路のような交わりの無い1次元的な領域のすべての点を警邏する
% 塀の警邏(Fence Patrolling)問題~\cite{chen2013fence, czyzowicz2011boundary}や,
% より一般的なグラフで辺全体ではなく頂点を警備する警邏問題~\cite{coene2011charlemagne},
% グラフと巡査が与えられて警邏可能かを判定する問題だけでなく,
% 塀の警邏問題においてなるべく長い塀を警邏する問題~\cite{czyzowicz2011boundary}や
% 全体の訪問の待ち時間の最大値を最小化する問題~\cite{chen2013fence}
% なども考えられている.
% \red{(加筆予定)}

また,{\graphLine}や{\graphStar}は木の特別な場合である.
