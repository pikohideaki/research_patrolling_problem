
\section{\unit}

% 完全グラフの場合は巡査が1人でもNP困難であることが示されていたので
定理\ref{theo:StarEqualProfitTimelimit}からStarの特殊な場合とみなせる{\unit}も
巡査1人で利得と{\timelimit}がすべて等しい場合は{\patrolling}がPであることがすぐに分かるが,
{\timelimit}さえすべて等しければ巡査数と利得が一般の場合でも{\patrolling}がPとなる.
%
これは,警備のコストとなる辺の長さと{\timelimit}の両方が全点で等しいことによって
単純に利得の大きい頂点から選べばよいためである.



\subsection{{\timelimit}がすべて等しい場合}




\subsection{{\timelimit}が一般の場合}



{\unit}で{\timelimit}がすべて等しい場合はPであることを示せたが,
{\timelimit}が一般の場合は多項式時間アルゴリズムやNP困難性を示すのが難しかったため,
ここでもLineのときのように,
最初の訪問時刻からその{\interval}ごとの時刻は必ず訪問しなければならないという問題をここでも考えてみる.



\red{
    1. 最初の訪問時刻も指定されるときの{\patrolling}は独立点集合問題からの帰着でNP困難. \\
    2. 全点警備可能性判定なら1人のときはP(おまけ). \\
    3. 最初の訪問時刻が指定されず自由度がある場合は
       Disjoint Residue Class Problem と同じ問題になるのでNP困難(おまけ).
}





\subsubsection{{\interval}の場合}
{\timelimit}を{\interval}に替えた問題では
巡査が1人の場合でも{\unit}上での{\decisionpp}がNP困難となる.

\begin{theo}
    \label{theo:Comp_exact_NPhard}
    グラフの形状が{\unit}で,
    {\interval}が与えられたときに,
    最初の訪問時刻からその{\interval}ごとの時刻は必ず訪問しなければならないという制約の場合,
    巡査が1人でも{\decisionpp}がNP困難である.
\end{theo}

\begin{proof}[証明]
    Disjoint Residue Class Problem~\cite{kawamura2015simple}からの帰着による.

    ある整数のペアの集合 $\mathset{ (m_1, r_1), \ldots, (m_n, r_n) }$ が
    Disjoint Residue Class であるとは,
    任意の整数 $x$ に対して $x \equiv r_i \mod m_i$ となるような $i$ が
    高々1つ存在することと定義される.
    Disjoint Residue Class Problem とは
    整数の組 $(m_1, \ldots, m_n)$ が与えられたときに,
    $\mathset{ (m_1, r_1), \ldots, (m_n, r_n) }$ が
    Disjoint Residue Class となるような組 $(r_1, \ldots, r_n)$ が
    存在するかを判定する問題であり,
    これはNP困難であることが知られている~\cite{kawamura2015simple}.

    Disjoint Residue Class Problem は,
    巡査1人,グラフの形状が{\unit}で
    {\interval}ごとの時刻は必ず訪問しなければならないという制約での{\decisionpp}に多項式時間帰着できる.
    Disjoint Residue Class Problem の入力が $(m_1, \ldots, m_n)$ のとき,
    {\unit}で頂点を$V = \mathset{ v_1, \ldots, v_n }$, {\interval}を$q_i = m_i$, 辺の長さを$d = 1$ とする.
    $d = 1$ より整数の時刻にいずれかの点を訪問できるようにする.
    各頂点 $v_i$ の最初の訪問時刻を $r_i$ とすると,
    この点を警備するために訪問しなければならない時刻の列は
    $q_i k + r_i (k \in \N)$ で与えられるが,
    全点を警備するためには任意の2点 $v_i, v_j \in V$ , 任意の整数 $k,l$ について
    $q_i k + r_i \neq q_j l + r_j$
    である必要がある.

    % Disjoint Residue Class Problem が Yes ならば \decisionpp も Yes
    Disjoint Residue Class Problem の解 $(r_1, \ldots, r_n)$ が存在するならば,
    これにより任意の時刻 $t \in \Z$ に対して $t \equiv r_i \mod q_i$, 
    すなわち $t = r_i + q_i k$ となる $k \in \Z$ が存在するような $i$ は高々1つであり,
    任意の $k,l \in \Z$ に対して $q_i k + r_i \neq q_j l + r_j$ が成り立つので,
    巡査は全点を警備できる.

    % \decisionpp が Yes ならば Disjoint Residue Class Problem も Yes
    逆に{\decisionpp}の解が存在するとき,
    全点を警備できるのでその警邏において各頂点$v_i$を最初に訪問する時刻を $r_i$ とすると,
    任意の $v_i,v_j \in V$, $k,l \in \Z$ に対し
    \begin{equation}
        \label{eq:patrollDisjoint}
        q_i k + r_i \neq q_j l + r_j
    \end{equation}
    が成り立つ.すると,任意の時刻 $t \in \Z$ に対して
    $t \equiv r_i \mod q_i$ となるような $i$ が2つ存在するとすると,
    それを $i$, $j$ として
    $t \equiv r_i \mod q_i$, $t \equiv r_j \mod q_j$
    すなわち,ある整数 $k,l$ が存在して
    $t = q_i k + r_i$, $t = q_j l + r_j$
    となり,この $k,l$ によって
    $q_i k + r_i = q_j l + r_j$ となり,式\ref{eq:patrollDisjoint} に矛盾する.
    よって,任意の整数 $t$ に対して $t \equiv r_i \mod q_i$ を満たす
    $i$ は高々1つであるような $r_i$ が与えられたので,
    {\decisionpp}の解が存在するとき,
    Disjoint Residue Class Problemにも解が存在する.

    以上よりDisjoint Residue Class Problemを帰着できた.
\end{proof}







\subsubsection{最初の訪問時刻指定,{\interval}}
今,最初の訪問時刻には自由度があり{\interval}だけが指定される問題を考えたが,
さらに最初の訪問時刻も与えられる問題も考えることができる.

\begin{theo}
    \label{theo:UnitSingleExactStarttime}
    グラフの形状が{\unit}で巡査が1人の場合,
    最初の訪問時刻と{\interval}が与えられて
    最初の訪問時刻からその{\interval}ごとの時刻は必ず訪問しなければならないという問題の場合,
    {\decisionpp}はPである.
\end{theo}

\begin{proof}[証明]
    まず,{\unit}の辺の長さを$d$とする.
    各$i$について正整数$q _i$と整数$r _i$とが与えられ,
        集合$S _i = \{\, q_i k + r_i : k \in \Z \,\}$に属する時刻に
        頂点$v _i$を訪問することが要求される.
    \red{(定義済?→)}{\decisionpp}がYesである(=全頂点を警備できる)ことは,
    連続した訪問しなければならない時刻の差がすべて移動時間$d$以上であること,すなわち
    任意の相異なる$i$,$j$について
        $S _i$に属するどの時刻と
        $S _j$に属するどの時刻も差が$d$以上であることを意味する.
    これは任意の2頂点$v_i, v_j \in V$と
    任意の整数$k,l$に対して$\abs{ (q_i k + r_i ) - (q_j l + r_j) } \geq d$という条件となり\red{(←「$(i, k) = (j, l)$の場合を除いて」が必要?)},これは
    任意の整数$n$に対して$\abs{ (r_i - r_j) + gcd(q_i, q_j) n } \geq d$と同値であり,
    左辺の最小値を考えると
    $\abs{ r_i - r_j }$を$gcd(q_i, q_j)$で割った余り$a$と
    $gcd(q_i, q_j) - a$ のうち小さい方が$d$以上かを計算すればよい.
    この計算は定数時間であり
    ${}_n C_2$ 通りこれを調べればよい.
\end{proof}




{\decisionpp}で巡査を複数とすると,$T$ に差が $d$ 未満の整数が含まれていても
複数の巡査によりそれぞれ訪問できる場合が生じるため難しい.



{\decisionpp}では巡査が1人ならば多項式時間アルゴリズムが存在したが,
一方{\patrolling}は巡査が1人でも(複数人でも)NP困難となる.

\begin{theo}
    \label{theo:UnitExactStarttimeNPhard}
    グラフの形状が{\unit}で,
    最初の訪問時刻と{\interval}が与えられたときに,
    最初の訪問時刻からその{\interval}ごとの時刻は必ず訪問しなければならないという問題の場合,
    巡査が1人で利得がすべて等しい場合でも{\patrolling}はNP困難である.
\end{theo}


\begin{proof}[証明]
    最大独立点集合問題からの帰着による.

    最大独立点集合問題は,
    無向グラフ$G = (V, E)$が与えられたときに
    独立点集合で最大のものを求める問題でNP完全であることが知られている.
    この問題において,間に辺の存在する2頂点の両方を選ぶことはできないという制約を,
    {\patrolling}において2頂点のどちらか一方しか警備できないという制約に帰着する.

    まず{\unit}の辺の長さを$d = 1$とする.
    これにより巡査がちょうど速さ$1$で動くとすると各頂点の訪問にかかる時間は$1$となり,
    {\unit}なのですべての整数の時刻にどれか1点を訪問できる.
    その上で,{\interval} $q_i$, 最初の訪問時刻 $r_i$ は整数,利得はすべて1とする.
    {\unit}の頂点集合は$V$とする.
    頂点$v_i$を警備するために訪問しなければならない時刻の列は
    $q_i k + r_i \; (k \in \N)$ で表される.
    すると,$v_i$ と $v_j$ の両方を警備できる必要十分条件は
    \[ q_i k + r_i = q_j l + r_j \]
    すなわち
    \[ r_i - r_j = q_j l - q_i k \]
    となる自然数 $k, l$ が存在しないこととなるが,
    これは $r_i - r_j = gcd(q_i,q_j) n$ となる整数 $n$ が存在しないことと同値である.
    よって,
    $v_i$ と $v_j$ の両方を警備できる必要十分条件は
    $r_i \not\equiv r_j \mod gcd(q_i, q_j)$
    となる.

    ここで,${}_n C_2$ 個の相異なる素数 $p_{ij} (1 \leq i < j \leq n)$ を用意し,
    各頂点の{\interval}を
    $q_i = p_{1i} p_{2i} \cdots p_{(i - 1)i} p_{i(i + 1)} \cdots p_{in}$
    とすると,$gcd(q_i,q_j) = p_{ij}$ ($i < j$ のとき)となり,
    先ほどの条件は
    $r_i \not\equiv r_j \mod p_{ij}$
    となる.

    $G$ において
    $(v_i, v_j) \in E$     ならば $r_i \equiv r_j \equiv 0 \mod p_{ij}$,
    $(v_i, v_j) \not\in E$ ならば $r_i \equiv 0, r_j \equiv 1 \mod p_{ij}$
    と定めると,
    各 $r_k$ に対して相異なる $n - 1$ 個の素数で割ったときの余りが与えられるので,
    中国剰余定理からそのような $r_k$ がその $n - 1$ 個の素数の積 $q_k$ を法として一意に存在することが言え,
    これにより,$(v_i, v_j) \in E$ ならば $r_i \equiv r_j \mod p_{ij}$
    $(v_i, v_j) \not\in E$ ならば $r_i \not\equiv r_j \mod p_{ij}$
    を満たすように各 $r_k$ を定めることができる

    最後に,${}_n C_2$ 個の相異なる素数を用意する計算量も確かめる必要がある.
    $k$ 番目に小さい素数を $P_k$ と書くと,$k \geq 6$ のときは
    $P_k < k( \ln k + \ln\ln k )$ であることが知られているため~\cite{dusart1999k},
    $k( \ln k + \ln\ln k )$ までの自然数を順に素数かどうか判定していくことで
    $k$ 個以上の素数を得ることができる.
    ある数が素数であるかどうかを判定する多項式時間アルゴリズムは存在するので~\cite{agrawal2004primes},
    ${}_n C_2$ 個の素数の列挙は$n$の多項式時間でできる.

    以上の手順で $q_k$ と $r_k$ を設定することにより,
    $G$ において最大の独立点集合を求める問題を,
    最初の訪問時刻が指定され周期ちょうど毎に訪問しなければならない問題で
    巡査が1人で利得がすべて等しい場合の{\patrolling}に帰着できた.
\end{proof}
