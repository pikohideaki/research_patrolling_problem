\section{{\unit}}
第1章で述べた通り,{\unit}はStarの特殊な場合とみなせるため,
定理\ref{theo:StarEqualProfitTimelimit}から
全点の利得と{\idletime}が等しい場合は
協力警邏問題を多項式時間で解くことができるが,
{\unit}の場合は全点の{\idletime}だけが等しければ協力警邏問題を多項式時間で解ける
(定理\ref{theo:UnitEqualTimelimit}).

{\idletime}が一般の場合については
多項式時間アルゴリズムやNP困難性を示すのが難しかったため,
第2章で扱った時刻指定協力警邏問題を再び考える.
グラフが{\unit}の場合は時刻指定協力警邏問題がNP困難になることを示す
(定理\ref{theo:unit_exacttime_NPhard}).



\subsection{全点の{\idletime}が等しい場合}

\begin{theo}
  \label{theo:UnitEqualTimelimit}
  グラフの形状が{\unit}で全点の{\idletime}が等しい場合,
  協力警邏問題は(利得,巡査数が一般であっても)多項式時間で解くことができる.
\end{theo}

\begin{proof}
  補題\ref{lemm:condition_of_guarding_star}から
  {\unit}の全点の許容訪問間隔が$Q$のとき,
  点集合$V$の任意の部分集合$W$について
  $$
    \sum_{v \in W} \min(d, Q) = |W|\min(d, Q) \leq mQ
    \iff \text{$m$人の巡査により$W$の全点を警邏できる}
  $$
  が成り立つ.$d$は{\unit}の各辺の長さである.

  グラフの形状が{\unit}の場合,
  全点の許容訪問間隔が等しいならば警邏する部分集合$W$は利得の大きい点から選べばよい
  (利得のより大きい点$v_1$とより小さい点$v_2$があるとき,
  $v_1$を警備して$v_2$を警備しない運行は常に$v_1$を警備する代わりに$v_2$を警備する運行に変換できる).
  $|W|\min(d, Q) \leq mQ$を満たす最大の$|W|$は
  $|W| = \left\lfloor {mQ}/{\min(d, Q)} \right\rfloor$
  であり,利得の大きい点から$\lfloor {mQ}/{\min(d, Q)} \rfloor$点を選ぶ計算は
  $O(\lfloor {mQ}/{\min(d, Q)} \rfloor \log n)$となる.
\end{proof}




\subsection{{\idletime}が一般の場合:時刻指定協力警邏問題}


\begin{theo}
  \label{theo:unit_exacttime_NPhard}
  グラフの形状が{\unit}のとき,
  時刻指定協力警邏問題は巡査が1人で全点の利得が等しくてもNP困難である.
\end{theo}

% 最大独立集合問題において,
% 無向グラフが与えられたときに独立点集合で最大のものを求めるが,
% 間に辺の存在する2頂点の両方を選ぶことはできないという制約を,
% 協力警邏問題において2頂点のどちらか一方しか警備できないという制約に変換する.

\begin{proof}[証明]
  最大独立集合問題からの帰着による.

  最大独立集合問題の入力のグラフが$G = (V, E)$のとき,
  時刻指定協力警邏問題に対して,
  巡査の人数$1$と{\unit}のグラフ$G' = (V', E')$を入力として与える.
  $G'$は以下のように定める.
  $V' = V$とし,全点の利得は$1$,辺の長さはすべて$1$とする.
  各点の訪問指定時刻は次のように定める.
  %
  まず,${}_n C_2$ 個の相異なる素数 $p_{i,j} (1 \leq i < j \leq n)$ を用意し,
  $q_i := \prod_{k = 1}^{i - 1} p_{1,k} \prod_{k = i + 1}^n p_{k,n}$
  とする.
  次に,$r_i (i \in \{1, \ldots, n\})$を,
  $G$のすべての2点$v_i, v_j (1 \leq i < j \leq n)$に対して,
  $(v_i, v_j) \in E$     ならば $r_i \equiv    r_j \equiv 0 \mod p_{i,j}$,
  $(v_i, v_j) \not\in E$ ならば $r_i \equiv 0, r_j \equiv 1 \mod p_{i,j}$,
  さらに$0 \leq r_i < q_i$
  を満たすように定める.
  各$r_i$に対して相異なる$n - 1$個の素数で割ったときの余りが与えられているので,
  中国剰余定理からそのような$r_i$がその$n - 1$個の素数の積$q_i$を法として一意に存在する.
  %
  以上のようにして得た$q_i, r_i (i \in \{ 1, \ldots, n\})$から,
  各点$v_i (i \in \{1, \ldots, n\})$の訪問指定時刻を$q_i k + r_i (k \in \Nset)$
  と定めると,
  時刻指定協力警邏問題の解は$G$の最大独立集合となる.

  実際,
  $G'$の任意の2点$v_i$, $v_j$($i < j$としてよい)に対し,
  この両方を警備できるための必要十分条件は,
  2点間の移動時間が$1$以上かかることから,
  訪問しなければならない時刻同士がすべて$1$以上離れていること,すなわち,
  任意の整数$k, l$に対し$|(k q_i + r_i) - (l q_j + r_j)| \geq 1$
  が成り立つこととなる.
  $q_i, r_i, q_j, r_j$がすべて整数のとき,これは
  $q_i k + r_i \neq q_j l + r_j$, 
  すなわち
  $r_i - r_j \neq q_j l - q_i k$
  が任意の整数$k, l$で成り立つこと同値である.
  $gcd(x,y)$を$x$と$y$の最大公約数とすると,
  $r_i - r_j \neq gcd(q_i,q_j) n$が任意の整数$n$で成り立つこと,
  つまり$r_i \not\equiv r_j \mod gcd(q_i, q_j) = p_{i,j}$と同値である.
  よって,
  $v_i$ と $v_j$ の両方を警備できる必要十分条件は
  $r_i \not\equiv r_j \mod p_{i, j}$
  となる.
  $r_i$の決め方から,
  \[
    (v_i, v_j) \in E \iff \text{$G'$の2点$v_i, v_j$を両方警備することができない}
  \]
  が成り立つため,
  $G'$の点部分集合であって同時に警備できない2点のうち少なくとも一方は含まないようなもののうち
  最大のものを選ぶと,
  これは$G$の最大独立集合となる.

  最後に,${}_n C_2$ 個の相異なる素数を用意する計算量も確かめる必要がある.
  $k$ 番目に小さい素数を $P_k$ と書くと,$k \geq 6$ のときは
  $P_k < k( \ln k + \ln\ln k )$ であることが知られているため~\cite{dusart1999k},
  $k( \ln k + \ln\ln k )$ までの自然数を順に素数かどうか判定していくことで
  $k$ 個以上の素数を得ることができる.
  ある数が素数であるかどうかを判定する多項式時間アルゴリズムは存在するので~\cite{agrawal2004primes},
  ${}_n C_2$ 個の素数の列挙は$n$の多項式時間でできる.
\end{proof}



定理\ref{theo:unit_exacttime_NPhard}では,
各点の訪問指定時刻といっても訪問間隔$q_i$と剰余$r_i$が与えられる場合について
NP困難性が示したが,
各点に訪問間隔$q_i$のみが指定されている問題も考えることができる.
この訪問間隔指定の協力警邏問題は,
全点警邏可能性判定であってもNP困難であることを示すことができる.



\begin{theo}
  \label{theo:NPhard_disjoint_residue_class_problem}
  グラフの形状が{\unit}のとき,
  訪問間隔指定の協力警邏問題は巡査が1人で全点警邏可能性判定であってもNP困難である.
\end{theo}


\begin{proof}[証明]
  Disjoint Residue Class Problem~\cite{kawamura2015simple}からの帰着による.

  ある整数の組の集合 $\mathset{ (m_1, r_1), \ldots, (m_n, r_n) }$ が
  Disjoint Residue Class であるとは,
  任意の整数 $x$ に対して $x \equiv r_i \mod m_i$ となるような $i$ が
  高々1つ存在することと定義される.
  Disjoint Residue Class Problem とは
  整数の組 $(m_1, \ldots, m_n)$ が与えられたときに,
  $\mathset{ (m_1, r_1), \ldots, (m_n, r_n) }$ が
  Disjoint Residue Class となるような組 $(r_1, \ldots, r_n)$ が
  存在するかを判定する問題であり,
  NP困難であることが知られている~\cite{kawamura2015simple}.

  Disjoint Residue Class Problem の入力が $(m_1, \ldots, m_n)$ のとき,
  訪問間隔指定協力警邏問題に対して
  巡査の人数$1$と$n$点からなる{\unit}のグラフで
  各点の訪問間隔を$q_i = m_i$となるように定め,辺の長さはすべて$1$としたものを
  入力として与えることで
  Disjoint Residue Class Problemを解くことができる.

  辺の長さが$1$であるから,すべての整数の時刻にいずれかの1点を訪問できる.
  各頂点$v_i$の最初の訪問時刻を$r_i$とすると,
  この点を警備するために訪問しなければならない時刻の列は
  $q_i k + r_i (k \in \N)$で与えられるが,
  全点を警備するためには任意の2点$v_i, v_j \in V$, 任意の整数$k,l$について
  $q_i k + r_i \neq q_j l + r_j$
  である必要がある.

  % Disjoint Residue Class Problem が Yes ならば \decisionpp も Yes
  Disjoint Residue Class Problem の解 $(r_1, \ldots, r_n)$ が存在するならば,
  任意の時刻 $t \in \Z$ に対して $t \equiv r_i \mod q_i$, 
  すなわち $t = r_i + q_i k$ となる $k \in \Z$ が存在するような $i$ は高々1つであり,
  任意の $k,l \in \Z$ に対して $q_i k + r_i \neq q_j l + r_j$ が成り立つので,
  巡査は全点を警備でき,
  解が存在しなければ
  あるの整数$x$に対して $x \equiv r_i \mod m_i$, $x \equiv r_j \mod m_j$
  となる$i, j$が存在するので,
  $v_i, v_j$を両方警備することができず,したがって全点を警備できない.

  % % \decisionpp が Yes ならば Disjoint Residue Class Problem も Yes
  % 逆に{\decisionpp}の解が存在するとき,
  % 全点を警備できるのでその警邏において各頂点$v_i$を最初に訪問する時刻を $r_i$ とすると,
  % 任意の $v_i,v_j \in V$, $k,l \in \Z$ に対し
  % \begin{equation}
  %     \label{eq:patrollDisjoint}
  %     q_i k + r_i \neq q_j l + r_j
  % \end{equation}
  % が成り立つ.すると,任意の時刻 $t \in \Z$ に対して
  % $t \equiv r_i \mod q_i$ となるような $i$ が2つ存在するとすると,
  % それを $i$, $j$ として
  % $t \equiv r_i \mod q_i$, $t \equiv r_j \mod q_j$
  % すなわち,ある整数 $k,l$ が存在して
  % $t = q_i k + r_i$, $t = q_j l + r_j$
  % となり,この $k,l$ によって
  % $q_i k + r_i = q_j l + r_j$ となり,式\ref{eq:patrollDisjoint} に矛盾する.
  % よって,任意の整数 $t$ に対して $t \equiv r_i \mod q_i$ を満たす
  % $i$ は高々1つであるような $r_i$ が与えられたので,
  % {\decisionpp}の解が存在するとき,
  % Disjoint Residue Class Problemにも解が存在する.

  % 以上よりDisjoint Residue Class Problemを帰着できた.
\end{proof}



% \subsubsection{最初の訪問時刻指定,{\interval}}
% 今,最初の訪問時刻には自由度があり{\interval}だけが指定される問題を考えたが,
% さらに最初の訪問時刻も与えられる問題も考えることができる.

% \begin{theo}
%     \label{theo:UnitSingleExactStarttime}
%     グラフの形状が{\unit}で巡査が1人の場合,
%     最初の訪問時刻と{\interval}が与えられて
%     最初の訪問時刻からその{\interval}ごとの時刻は必ず訪問しなければならないという問題の場合,
%     {\decisionpp}はPである.
% \end{theo}

% \begin{proof}[証明]
%     まず,{\unit}の辺の長さを$d$とする.
%     各$i$について正整数$q _i$と整数$r _i$とが与えられ,
%         集合$S _i = \{\, q_i k + r_i : k \in \Z \,\}$に属する時刻に
%         頂点$v _i$を訪問することが要求される.
%     \red{(定義済?→)}{\decisionpp}がYesである(=全点を警備できる)ことは,
%     連続した訪問しなければならない時刻の差がすべて移動時間$d$以上であること,すなわち
%     任意の相異なる$i$,$j$について
%         $S _i$に属するどの時刻と
%         $S _j$に属するどの時刻も差が$d$以上であることを意味する.
%     これは任意の2頂点$v_i, v_j \in V$と
%     任意の整数$k,l$に対して$\abs{ (q_i k + r_i ) - (q_j l + r_j) } \geq d$という条件となり\red{(←「$(i, k) = (j, l)$の場合を除いて」が必要?)},これは
%     任意の整数$n$に対して$\abs{ (r_i - r_j) + gcd(q_i, q_j) n } \geq d$と同値であり,
%     左辺の最小値を考えると
%     $\abs{ r_i - r_j }$を$gcd(q_i, q_j)$で割った余り$a$と
%     $gcd(q_i, q_j) - a$ のうち小さい方が$d$以上かを計算すればよい.
%     この計算は定数時間であり
%     ${}_n C_2$ 通りこれを調べればよい.
% \end{proof}


% {\decisionpp}で巡査を複数とすると,$T$ に差が $d$ 未満の整数が含まれていても
% 複数の巡査によりそれぞれ訪問できる場合が生じるため難しい.
