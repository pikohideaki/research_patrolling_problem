\begin{abstract}
1人または複数の巡査が所与の領域を動き回り,その領域内のあらゆる場所を十分な頻度で訪問することで,これを守備,監督することを警邏という.

Coeneらは,辺の長さの与えられた無向グラフと各頂点の持つ利得・許容訪問間隔,巡査の人数が与えられたとき,グラフの辺上を速さ1以下で動く1人または複数の巡査が,各頂点を訪問する間隔をその頂点の許容訪問間隔以内にしなければならない制約を満たしながら訪問し続けることで警備するとき,警備される頂点の利得の合計を最大化するような頂点部分集合を求める警邏問題を考えた.
しかし,Coeneらは巡査が複数の場合もどの頂点も高々1人の巡査により警備されるという仮定を加えていた.

本稿ではこの問題を取り上げ,複数の巡査が協力して警備する頂点があってもよいという設定で同じ問題を考える.
NP困難性や多項式時間アルゴリズムを示すのが難しかった部分については,許容訪問間隔の代わりに「厳密訪問間隔」を考え,頂点を最初に訪問した時刻からちょうど厳密訪問間隔ごとの時刻にその頂点は訪問されなければならないという問題や,さらに最初の訪問時刻も与えられる問題を考えて計算量クラスの評価を試みる.


% Coeneらは,
% 辺の長さの与えられた無向グラフと各頂点の持つ利得・許容訪問間隔,巡査の人数が与えられたとき,
% 巡査がグラフ上を速さ1以下で動き頂点を訪問することで
% 警備できる頂点集合のうち,利得の合計を最大化するものを求める警邏問題を考えた.
% ある頂点を警備できているとは,
% どの連続した2回の訪問も間隔がその点の許容訪問間隔以内となるように
% 訪問し続けられることと定義される.
% グラフが
% Line(1つの線分上にすべての頂点があるグラフ)または
% Circle(Line の両端をつなぐ辺を足したグラフ)
% で巡査が1人の場合には多項式時間アルゴリズム,
% 星,木,一般の場合についてはNP困難性が示されている.
% 本稿ではこの問題を取り上げ,
% 多項式時間で解けるか否か未解決であった
% Line で巡査が複数の場合や,
% 既にNP困難性が示されている星よりも単純な,
% 星で枝の長さがすべて等しい図形(本稿では UStar と呼ぶことにする)について
% 詳しく調べることにする.
% そのままの問題設定では解決できなかった部分については,
% 許容訪問間隔の代わりに厳密訪問間隔が与えられたときに,
% 最初の訪問時刻からその厳密訪問間隔ごとの時刻は必ず訪問しなければならないという問題,
% さらに最初の訪問時刻も与えられる問題も考える.
\end{abstract}
