\subsection*{関連研究}
\ncomment{[文章化,引用の追加]}

% 与えられた領域内に幾人かの巡査をうまく配置することを目的とする問題
警邏に関する問題には様々な設定が考えられている.

[メモ]
\begin{itemize}

  \item 問題設定の大枠について
  \begin{itemize}
    \item 現実世界の平面上を動く巡査の警備のモデル化として,
      警備対象が存在する部屋の周囲を巡回するものや,
      部屋の内部を巡査が動き回るものなどが考えられる.
    \begin{itemize}
      \item 
        前者は
        「塀の警邏問題(fence patrolling)」
        ・「境界の警邏問題(boundary patrolling)
        などの名前で知られており,
        塀を表す線分または閉路のすべての点を警備対象とするもの~\cite{czyzowicz2011boundary, dumitrescu2014fence, kawamura2015fence},
        より一般に線分や閉路の一部のみを警備対象とする問題などが考えられている~\cite{collins2013optimal}.
        % 警備対象を有限個の点とすると{\patProb}と似た問題になり,警備対象を全体とすると塀の警邏問題と似た問題となる.
      \item 
        部屋の内部を警備する場合には部屋の内部には複数の障害物が存在する状況が考えられる~\cite{}.
        
    \end{itemize}
    \item {\patProb}では,{\maxIdletime}の条件さえ満たしていれば警備できるという設定で考えており,侵略者の動きなどは具体的に考えていないが,侵略者の動き方まで含めてゲーム理論的に考察している研究も存在する~\cite{brazdil2015strategy, papadaki2016patrolling}.
    \item {\patProb}では,巡査達の動きを決定論的に与えるので,
    モデル化したい現実の状況によっては十分賢い侵略者に対応されてしまい有用でない場合がありうる.
    このような欠点を補うため,巡査の運行にランダム性を取り入れたものがある~\cite{}.
    \item {\patProb}では,警備対象の環境が与えられたときに最適な巡査の運行を最初に決定してしまって,巡査がその通りに動くという意味で,中央集権的な運行の決定の仕方である.
    一方で,巡査がその近傍の情報から各々の判断で運行を決定するという設定のものも考えられている~\cite{}.
  \end{itemize}

  \item 問題に対する様々なアプローチ
  \begin{itemize}
    \item 理論的な研究を行うもの~\cite{}
    \item ヒューリスティックな戦略を先に与え,計算機でシミュレーションを行うもの~\cite{}
    \item 実際のロボットで実験しているものもある~\cite{}.
  \end{itemize}

  \item {\patProb}を考える動機
  \begin{itemize}
    \item 巡査が障害物を含む2次元平面内を動き回り警備するという目的の問題において,障害物を含む2次元平面をグラフに簡略化して考えるというところから,グラフの頂点を警備するという問題が考えられていた~\cite{machado2002multi}.
    \item (なぜ図形として{\graphLine}, {\graphStar}, {\graphUnit}を考えたか?)
    → {\graphLine}は塀の警邏などの文脈でよく現れるため~\cite{}.{\graphStar}は木の特殊な場合であり,グラフが木の場合の{\patProb}の困難性の評価に有用な図形である.グラフの頂点の警邏を行うために,そのグラフの最小全域木を計算し代わりにこれを警邏するとする研究もある~\cite{}\ncomment{[確認]}.
  \end{itemize}

  \item 似た問題設定の他の研究との細かい違いについて
  \begin{itemize}
    \item 目標(いずれも与えられた$m$人の巡査で全点を警邏できるか判定する問題の一般化)
    \begin{itemize}
      \item 全点警備可能な最小の巡査数を求める~\cite{}
      \item 与えられた巡査により警邏可能な部分集合であって点の数や利得の合計が最大のものを求める~\cite{}
      \item 与えられた巡査により全点を警備する上で,
      各点の訪問頻度を(同程度にする・平均値を最大化する・最小の訪問頻度を最大化する)~\cite{}
    \end{itemize}
  \end{itemize}


  \item 拡張について
  \begin{itemize}
    \item
      我々の考える{\patProb}は,非常に単純な図形についてさえNP困難性が示されていたり,多項式時間アルゴリズムを示すことができていない場合が存在する.
      よって,{\patProb}に対して問題設定の拡張を考えること自体はできても
      厳密な最適解を得る効率的なアルゴリズムは望みがたい.
    \item 例えば,巡査の速さの上限が異なる場合などを考えることもできるが,
      グラフの警邏の場合については・・・\ncomment{[調べる]}.
      塀の警邏問題について
      速さの上限が異なる巡査の場合について調べている研究はいくつか存在する~\cite{}
      が,一般の状況に対する最適解を与えることには成功していないようである~\cite{}\ncomment{[確認]}.
    \item {\patProb}は頂点を通過する(時間$0$滞在する)だけで警備したことになる設定だが,
      より一般に時間$d \geq 0$滞在しなければならないという拡張が考えられる.
      しかしこれについては
      点$p$を警備するのに時間$d$滞在しなければならない状況を,
      $p$から長さ$d/2$の辺を伸ばした先の点$p'$を代わりに警備対象とし滞在時間$0$とするグラフに帰着できるため
      拡張にはなっていない.
    \item 巡査が視野を持つ

  \end{itemize}

  \item その他
  \begin{itemize}
    \item 分割戦略と巡回戦略
    \begin{itemize}
      \item 多くの場合TSPの解に基づく巡回戦略が最適だが,
        長い辺がある場合は分割戦略が有利であり,
        また巨大グラフの場合にはTSPの解の計算コストが大きくなる問題がある.
      \item TSPの解に基づく巡回戦略が多くの場合最適なので,TSPの解を求める近似アルゴリズムにより巡査の運行を決めるものもある.
      グラフを最小全域木に簡単化し,この上でTSPの解を求め巡回戦略を与えるという近似をしているものがある.
      → グラフが木の場合の多項式時間アルゴリズムやNP困難性の証明には意味がある.
    \end{itemize} 
    % \item 他にも,警備対象や巡査の動く領域が時間変化するような場合なども現実の模倣としては考えられるが,
    % 簡単化されたモデルを扱うものが多い.
  \end{itemize}
\end{itemize}


% より一般的なグラフで辺全体ではなく頂点を警備する警邏問題~\cite{coene2011charlemagne},
% グラフと巡査が与えられて警邏可能かを判定する問題だけでなく,
% 塀の警邏問題においてなるべく長い塀を警邏する問題~\cite{czyzowicz2011boundary}や
% 全体の訪問の待ち時間の最大値を最小化する問題~\cite{chen2013fence}
% なども考えられている.

% また,{\graphLine}や{\graphStar}は木の特別な場合である.
