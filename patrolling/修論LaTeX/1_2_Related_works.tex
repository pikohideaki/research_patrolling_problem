\subsection*{関連研究}
% 与えられた領域内に幾人かの巡査をうまく配置することを目的とする問題
警邏に関する問題には様々な設定が考えられている.

[メモ]
\begin{itemize}
  \item {\patProb}を考える動機
  \begin{itemize}
    \item 巡査が障害物を含む2次元平面内を動き回り警備するという目的の問題において,障害物を含む2次元平面をグラフに簡略化して考えるというところから,グラフの頂点を警備するという問題が考えられていた~\cite{machado2002multi}.
    \item (なぜ図形として{\graphLine}, {\graphStar}, {\graphUnit}を考えたか?) → {\graphLine}は塀の警邏などの文脈でよく現れるため.{\graphStar}は高さが$1$の木であり,グラフが木である場合の{\patProb}の困難性の評価に有用な図形である.
    \item (なぜ{\maxIdletime}を警備の条件にしたか?)→ Coeneらの先行研究と似た設定を考えたかったため(自然な条件なのであまり説明しなくてよさそう?)
    \item 頂点を通過するだけで警備したことになる設定だが,より一般に時間$d \geq 0$滞在しなければならないとしたらどうか? → 点$p$を警備するのに時間$d$滞在しなければならないという条件は,$p$から長さ$d/2$の辺を伸ばした先の点$p'$を代わりに警備対象とするグラフを入力とする{\patProb}に帰着できる.
    \item 
    \item 
  \end{itemize}

  \item 問題設定の大枠について
  \begin{itemize}
    \item 現実の警備の問題としては,警備の仕方として領域内を巡査動き回るというものの他にも,領域の周囲を巡回するという警備の仕方も考えられる.実際に塀の警邏問題として知られている~\cite{czyzowicz2011boundary, dumitrescu2014fence, kawamura2015fence}.
    \begin{itemize}
      \item 一次元の連続領域のすべての点が警備対象
      \item 線分や閉路の一部のみが警備対象であるという中間的な問題設定も考えられている~\cite{collins2013optimal}.警備対象を有限個の点とすると{\patProb}と似た問題になり,警備対象を全体とすると塀の警邏問題と似た問題となる.
    \end{itemize}
    \item {\patProb}では,{\maxIdletime}の条件さえ満たしていれば警備できるという設定で考えており,侵略者の動きなどは具体的に考えていないが,侵略者の動き方まで含めてゲーム理論的に考察している研究も存在する~\cite{brazdil2015strategy, papadaki2016patrolling}.
    \item {\patProb}では,巡査達は決定論的に動くので,現実的には十分賢い侵略者に対応されてしまう欠点がある.
    巡査の運行にランダム性を取り入れたものがある~\cite{}.
    \item {\patProb}では,警備対象の環境が与えられたときに最適な巡査の運行を決定してしまって巡査がその通りに動くという意味で,中央集権的な運行の決定をしているとみなすことができる.
    一方で,巡査がその近傍の情報から各々の判断で運行を決定するという設定のものも考えられている~\cite{}.
  \end{itemize}

  \item 問題に対する様々なアプローチ
  \begin{itemize}
    \item 理論的な研究を行うもの
    \item ヒューリスティックな戦略を先に与え,計算機でシミュレーションを行うもの
    \item 実際のロボットで実験しているものもある
  \end{itemize}

  \item 似た問題設定の他の研究との細かい違いについて
  \begin{itemize}
    \item 最適化する指標
    \begin{itemize}
      \item 全点警備可能な最小の巡査数を求める
      \item 警邏できる部分集合であって点の数や利得の合計が最大のものを求める
      \item 与えられた巡査により全点を警備する上で,
      各点の訪問頻度を(同程度にする・平均値を最大化する・最小の訪問頻度を最大化する)など
      \begin{itemize}
        \item 
      \end{itemize}
    \end{itemize}
    \item 巡査の速さがすべて同じであるという仮定を置いているが,異なる速さの巡査が存在する場合も調べられている~\cite{}.
  \end{itemize}

  \item その他
  \begin{itemize}
    \item 分割戦略と巡回戦略
    \begin{itemize}
      \item 多くの場合TSPの解に基づく巡回が最適だが,
      長い辺がある場合や巨大グラフの場合に問題がある.
      \item TSPの解に基づく巡回が多くの場合最適なのでTSPの解を求める近似アルゴリズムにより巡査の運行を決めるものも
    \end{itemize} 
    % \item 他にも,警備対象や巡査の動く領域が時間変化するような場合なども現実の模倣としては考えられるが,
    % 簡単化されたモデルを扱うものが多い.
    % \item {\patProb}は一般グラフではハミルトン閉路問題の帰着によりNP困難であることが知られているが,
    % 逆にTSPの解による巡回で
  \end{itemize}
\end{itemize}


% 警邏に関する研究には様々な問題設定があり,
% 例えば線分や閉路のような交わりの無い1次元的な領域のすべての点を警邏する
% 塀の警邏(Fence Patrolling)問題~\cite{chen2013fence, czyzowicz2011boundary}や,
% より一般的なグラフで辺全体ではなく頂点を警備する警邏問題~\cite{coene2011charlemagne},
% グラフと巡査が与えられて警邏可能かを判定する問題だけでなく,
% 塀の警邏問題においてなるべく長い塀を警邏する問題~\cite{czyzowicz2011boundary}や
% 全体の訪問の待ち時間の最大値を最小化する問題~\cite{chen2013fence}
% なども考えられている.

% また,{\graphLine}や{\graphStar}は木の特別な場合である.
