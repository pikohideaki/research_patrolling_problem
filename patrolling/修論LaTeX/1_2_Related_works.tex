\section{関連研究}
\label{section: relatedWorks}

\ncomment{[ToDo: 引用の追加]}

\ncomment{[編集中]}

警邏に関する問題には様々なものが考えられている.
モデル化としては警邏の問題と似ているが
警邏とは異なる現実の問題を想定している研究もあり,
機械の定期的な保守・整備\cite{anily1998scheduling, bar2002minimizing},
「周期的スケジューリング問題(periodic scheduling problem)」\cite{sgall2009periodic},
配達\cite{campbell2005vehicle},
「定期的な待ち時間の問題(periodic latency problem)」\cite{coene2011charlemagne}
など,様々な名称で論じられている.


警邏という目的に限っても,モデル化には様々なものがある.


我々の{\patProb}ではグラフ上の有限個の点を警邏するが,
グラフの辺を警邏する問題もある\cite{yanovski2003distributed}.
% \cite{yanovski2003distributed}では無向グラフのすべての辺上の点をなるべく同じ頻度で訪問することを目標としている.

グラフの辺を警邏する問題に関連して,グラフの形状を限ったものとして
「塀の警邏(fence patrolling)」や「境界の警邏(boundary patrolling)」
などの名前で知られている問題がある\cite{
    czyzowicz2011boundary,
    chen2013fence,
    elmaliach2008realistic,
    collins2013optimal}.
塀の警邏問題では線分や閉路で表される塀に沿って巡査が両方向または一方向に動く.
塀全体の点に対する訪問の待ち時間の最大値を最小化するもの\cite{chen2013fence},
塀の任意の点をある時間以上放置せずに訪問する条件の下で
なるべく長い塀を警邏するもの\cite{czyzowicz2011boundary}など,
様々な目標が考えられている.
%
Elmaliachら\cite{elmaliach2008realistic}は塀が線分の場合について,
各警邏対象を訪問する頻度をなるべく同程度にする,
各警備対象を訪問する頻度の平均値を最大化する,
警備対象のうち訪問頻度が最小のものを最大化する,
という3つの目標について論じている.
%
このように,警邏には様々な目標が考えられている.
%
塀の警邏については巡査の速さの最大値が異なる場合も考えられている
\cite{czyzowicz2011boundary, czyzowicz2016fence, kawamura2015fence}が,
任意の巡査数に対する最適な運行は分かっていない.
%
線分や閉路のすべての点を警邏対象とするのではなく,
線分や閉路に含まれる有限個の区間の合併のみを警邏対象とするという拡張も考えられている\cite{collins2013optimal}.
特に,この\cite{collins2013optimal}の問題は警邏対象の区間を有限個の点とすると
我々の{\patProb}の{\graphLine}の場合と似た問題になり,関係が深い.
%
以上はいずれも,巡査は点で表されており,塀の点の訪問とは巡査がその位置を通ることであった.
これに対し,Czyzowiczら\cite{czyzowicz2014patrolling}は
巡査が一点より広い視野を有するという拡張について調べている.



%%% 美術館問題
警邏ではなくなるが,
部屋の内部を監視するという目的の問題としては美術館問題\cite{lee1986computational}という問題もある.
美術館問題では,
部屋にいくつかの防犯カメラを設置し全体を監視するという目標を考える.
部屋が多角形$P$で表されるとき,
防犯カメラの位置を表す点集合$G$であって
$P$の任意の点が$G$のいずれかの点から見えるようなもののうち
最小のものを求めるという問題である.
$P$のある点$p$が点$g$から見えるとは,
$p$と$g$を結ぶ線分の全体が$P$に含まれることである.
美術館問題はNP困難であることが知られている\cite{lee1986computational}.
%
固定された防犯カメラの位置を決める問題でさえNP困難であることからも,
二次元の領域を動く巡査の最適な運行は望みがたく,
巡査の動く領域をグラフ上に制限する,
視野を制限する,巡査の人数が少ない場合を考えるなど,
なんらかの制約を設ける必要があると予想される.

Machadoら\cite{machado2002multi}は,
障害物を含む二次元平面で表される領域を"skeltonization"によりグラフにし,
グラフの頂点を警邏する問題としてモデル化している.
\cite{machado2002multi}では,
辺の長さはすべて$1$,
巡査の視野はその存在する位置の一点,
すべての頂点は同じ優先度とするなどの制約を設け,
頂点の放置時間の最大値を最小化するなどいくつかの目標について
ヒューリスティックな巡査の運行戦略をいくつか考え計算機実験により評価している.
%
グラフの頂点を警邏する問題を扱っている点で我々の{\patProb}と関係が深いが,
我々は巡査の最適な運行を求める問題の計算量クラスを理論的に調べることを目的としている点で異なる.


警邏の問題を侵略者の動きまで含めてゲーム理論的に考察している研究もある\cite{
    alpern2017security,
    alpern2016patrolling,
    alpern2011patrolling,
    brazdil2015strategy,
    fomin2008annotated,
    garrec2016continuous,
    papadaki2016patrolling}.
Br{\'a}zdilら\cite{brazdil2015strategy}の問題では,
巡査と侵略者は1人ずつおり,それぞれあるグラフ$G = (V, E)$上を動く状況を考える.
$T \subseteq V$が定められており,侵略者は$T$の点を訪問することを目指すが,
点に巡査が存在するときに侵略者も同時にその点を訪問することはできない.
巡査は単位時間に一つ辺を渡るが,どの辺を選択し渡るかは前回の選択とは独立に確率的に決まる.
巡査の戦略とは訪問する点の列に対する確率のことである.
侵略者は巡査の位置や戦略まで知っているが,次にどの辺を選択し移動するかまでは予測できない.
%
{\patProb}では,点$v$は{\maxIdletime}の条件さえ満たしていれば警邏されるとしており
具体的な侵略者のことは考えていないので,
最適な巡査の運行を決定論的に定めると
巡査の位置を予測できる侵略者に対応されてしまう可能性がある.
Br{\'a}zdilら\cite{brazdil2015strategy}の問題では巡査が確率的に動くことで
そのような弱点を補っているとみなすこともできるだろう.

巡査がその近傍の情報から各々の判断で運行を決定するという戦略も考えられており\cite{santana2004multi},
% {\patProb}では,警邏対象の環境が与えられたときに最適な巡査の運行を最初に決定してしまって巡査がその通りに動くという中央集権的な運行の決定の仕方の弱点を補うことができると思われる.




% これまで挙げた警邏の問題では,
% 複数の巡査の速さの異なる場合,
% 巡査が確率的に動く場合,
% 巡査が視野を持つ場合など,
% 複雑な設定を考えているときには代わりに巡査は一人とする,領域を制限するなど,
% 他の部分を簡単にするかまたは
% 厳密解を諦め近似や計算機実験を行うことが多いことが分かる.


{\patProb}にはいくつかの拡張が考えられる.

{\patProb}は,非常に単純な図形についてさえNP困難性が示されていたり(\cite[Theorems 5 and 6]{coene2011charlemagne}),
多項式時間アルゴリズムを示すことができていない場合(地図が{\graphLine}で巡査が複数の場合など)が存在する
ように,
既に複雑な問題設定である.
よって,{\patProb}に対して問題設定の拡張を考えること自体はできても
厳密な最適解を得る効率的なアルゴリズムは望みがたい.

例えば,巡査の速さの最大値が異なる場合などが考えられる.
しかし,塀(線分)の警邏で巡査の速さの最大値が異なる場合について
複雑な運行が生じる例が知られており\cite{kawamura2015fence},
また任意の巡査数に対する最適な運行は分かっていないことからも,
{\patProb}で巡査の速さの最大値が異なる場合も同様に難しいと予想される.

\newcommand{\patProbWithDuration}{{滞在時間付き{\patProb}}}
他には,
{\patProb}ではある点を訪問するには通過する(時間$0$滞在する)だけでよいとしているが,各点を訪問するには時間$s \geq 0$滞在しなければならないという
条件を追加した{\patProbWithDuration}も考えられる.
Coeneら\cite{coene2011charlemagne}はこの問題が,
すべての点が同じ位置にあり巡査が一人で全点を警邏できるかの判定であってもNP困難であることを示している\cite[Theorem~3]{coene2011charlemagne}.
%
{\patProbWithDuration}は一見複雑な拡張に思われるが,
実は,{\patProbWithDuration}は次のように{\patProb}に帰着することができる.
{\patProbWithDuration}の入力が
地図$M = (U, V)$と
$V$の各点$v_i\ (i \in \{ 1, \ldots, n \})$の必要滞在時間$s_i \geq 0$であるとする.
各$v_i\ (i \in \{ 1, \ldots, n \})$について,
$v_i$から長さ$s_i/2$の辺$e_i \not\subseteq U$を伸ばした先の点を$v_i'$とする.
$v_i' \notin U$とする.
$V' = \{ v_1', \ldots, v_n' \}$,
$U' = U \cup \bigcup_{i \in \{ 1, \ldots n \}} e_i$とする.
{\patProb}に地図$M' = (U', V')$を入力として与えればよい.
%
こうすることで,
$U'$の点$v_i'$を訪問するには長さ$s_i/2$の辺$e_i$を通らねばならず$v_i$と$v_i'$の間の往復には$s_i$の時間を要するので,
$v_i$に時間$s_i$滞在するのと同じ状況を作ることができる.

上の帰着では,
すべての点が同じ位置にある地図で滞在時間が付いたものは{\graphStar}に変換される.
{\graphStar}の地図を巡査一人で全点を警邏できるかの判定はNP困難である\cite[Theorem~10]{coene2011charlemagne}ので,
{\patProbWithDuration}はすべての点が同じ位置にある地図で巡査が一人で全点を警邏できるかの判定であってもNP困難であることが分かる.
%
また,{\graphLine}の地図で滞在時間が付いたものは
図\ref{fig: }のような二分木に変換される.
地図が木で巡査が一人かつ全点の利得と{\maxIdletime}が等しい場合は
{\patProb}は多項式時間で解ける\cite[Corollary~3]{coene2011charlemagne}ので,
地図が{\graphLine}で巡査が一人かつ全点の利得と{\maxIdletime}が等しい場合は
{\patProbWithDuration}は多項式時間で解ける.

{\patProb}によって滞在時間付きの場合を考えることができることから,
{\patProb}は周期的スケジューリング問題(periodic scheduling problem)\cite{liu1973scheduling}とも関係する.
LiuとLayland\cite{liu1973scheduling}は~~~という問題を扱っているが,
これは地図が{\graphUnit}で二点間距離が$1$の場合の{\patProbWithDuration}と同じ問題となる.


% \ref{chapter: introduction}でも述べた通り,
% {\patProb}は巡査が一人かつ
% 全点の利得と{\maxIdletime}が等しい場合に限っても
% NP困難である\cite[Theorem~8]{coene2011charlemagne}.
我々は地図の形状として
{\graphLine}, {\graphStar}, {\graphUnit}%
を調べたが,
これらはCoeneら\cite{coene2011charlemagne}が
扱っていた形状の一部である.
Coeneらはさらに
地図がCircle(閉路),{\graphTree}(木)の場合についても調べている\cite{coene2011charlemagne}.
Circleは{\graphLine}を,{\graphTree}は{\graphLine},{\graphStar},{\graphUnit}を
それぞれ特殊な場合として含んでいる.
%
地図が{\graphLine}のときの{\patProbWithDuration}が
地図が{\graphTree}のときの{\patProb}に帰着されることや,
一般のグラフの頂点の警邏において
最小全域木を代わりに警邏するという近似も考えられている\cite{}ことからも,
{\graphTree}は重要である.
地図が{\graphTree}の場合の{\patProb}については\ref{chapter: star}の最後に言及する.


% \begin{itemize}
%   \item 分割戦略と巡回戦略
%   \begin{itemize}
%     \item 多くの場合TSPの解に基づく巡回戦略が最適だが,
%       長い辺がある場合は分割戦略が有利であり,
%       また巨大グラフの場合にはTSPの解の計算コストが大きくなる問題がある.
%     \item TSPの解に基づく巡回戦略が多くの場合最適なので,TSPの解を求める近似アルゴリズムにより巡査の運行を決めるものもある.
%     グラフを最小全域木に簡単化し,この上でTSPの解を求め巡回戦略を与えるという近似をしているものがある.
%     → グラフが木の場合の多項式時間アルゴリズムやNP困難性を示すのは重要である.
%   \end{itemize} 
% \end{itemize}

