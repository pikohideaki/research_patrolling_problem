\section*{木の場合}

{\graphStar}を一般化した地図の形状として木を考えることができる.
本節ではこれを{\graphTree}と呼び,地図が{\graphTree}の場合の{\patProb}について考える.
{\graphTree}では各葉$v$のみに{\maxIdletime}が定められているとする.

地図が{\graphTree}の場合,{\patProb}は,
{\graphTree}の特殊な場合である{\graphStar}の場合の結果から,
利得が一般の場合や{\maxIdletime}が一般の場合は巡査が一人であってもNP困難である.
よって以降は,全点の利得と{\maxIdletime}が等しい場合を考える.

地図が{\graphTree}で巡査が一人の場合,{\patProb}は多項式時間で解けることが
示されている
\cite[Corollary 3]{coene2011charlemagne}
(Coeneら\cite{coene2011charlemagne}の{\independentPatProb}についての結果だが,
巡査が一人の場合は{\patProb}と同じ問題となる).
そこで,我々は巡査数が一般の場合を調べたい.
なお,
Coeneら\cite{coene2011charlemagne}の{\independentPatProb}では,
地図が{\graphTree}で巡査数が一般の場合はNP困難である.

\ncomment{加筆}

まず例として,図\ref{fig: }\ncomment{[図追加]}のような
2つの星の中心どうしが一本の橋で結ばれた木について考える.
この例では,2つの星はそれぞれ全点警邏に2人の巡査を要する
(補題\ref{lemm:StarConditionOfGuarding})が,
木全体を図\ref{fig: }のように巡回すると3人の巡査で全点を警邏できる.




\ncomment{加筆}

地図が{\graphTree}の場合も多項式時間アルゴリズムが存在すると予想しているが,未解決である.
