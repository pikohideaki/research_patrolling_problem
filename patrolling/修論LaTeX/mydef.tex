\def\newblock{\hskip .11em plus .33em minus .07em}

\newcommand{\figdir}{../figures}
% \newcommand{\ncomment}[1]{{\textcolor{red}{#1}}}
% \newcommand{\kcomment}[1]{{\textcolor{blue}{#1}}}

% \definecolor{mygreen}{rgb}{0, 0.55, 0}
% \newcommand{\temporaryname}[1]{\textbf{\color[rgb]{0, 0.55, 0}{#1}}}

% 数学記号
\newcommand{\Rset}{\mathbf R}
\newcommand{\Nset}{\mathbf N}
\newcommand{\Zset}{\mathbf Z}
\newcommand{\abs}[1]{\lvert{#1}\rvert}
\newcommand{\card}[1]{\lvert{#1}\rvert}

\newcommand{\defword}[1]{\textbf{#1}}

% 警邏問題名称
\newcommand{\graphLine}{Line}
\newcommand{\graphStar}{Star}
\newcommand{\graphUnit}{Unit}
\newcommand{\graphTree}{Tree}
\newcommand{\maxIdletime}   {放置限度}
\newcommand{\exactTime}     {指定訪問時刻}
\newcommand{\exactInterval} {指定訪問間隔}
\newcommand{\PP}                    {全点警邏問題}
\newcommand{\PPProfit}              {警邏問題}
\newcommand{\independentPP}         {独立警邏問題}
\newcommand{\timeSpecifiedPP}       {全点定時訪問問題}
\newcommand{\timeSpecifiedPPProfit} {定時訪問問題}
\newcommand{\intervalSpecifiedPP}   {全点定期訪問問題}
\newcommand{\setPartAlgo}{最小運行可能分割アルゴリズム}


%%%%% 定理環境 %%%%%
\usepackage{amsmath,amssymb,amsthm}
\makeatletter
\renewenvironment{proof}[1][\proofname]{\par
  \pushQED{\qed}%
  \normalfont \topsep.5\baselineskip \labelsep1zw
  \trivlist
  \item[\hskip\labelsep
        \textsf{#1}]\ignorespaces
}{%
  \popQED\endtrivlist\@endpefalse
}
\makeatother
\renewcommand{\proofname}{証明}
\newtheoremstyle{jthm}{.5\baselineskip}{.5\baselineskip}{\normalfont}{0zw}{\headfont}{}{1zw}{\thmname{#1}\thmnumber{#2}\thmnote{【#3】\inhibitglue}}
\theoremstyle{jthm}
\newtheorem {theo}      {定理}[chapter]
\newtheorem {defi}[theo]{定義}
\newtheorem {lemm}[theo]{補題}

\newtheorem*{patrollingProblem}{{\PPProfit}}
\newtheorem*{timeSpecifiedPatrollingProblem}{{\timeSpecifiedPPProfit}}
\newtheorem*{intervalSpecifiedPatrollingProblem}{{\intervalSpecifiedPP}}
\newtheorem*{setPartitionAlgorithm}{{\setPartAlgo}}
