\chapter{{\graphUnit}}
\label{chapter: unit}

\ref{chapter: introduction}節で述べたように
{\graphUnit}は{\graphStar}の特殊な場合とみなせるため,
全点の利得と{\maxIdletime}が等しい場合,{\patProb}は
定理\ref{theo:StarUnaryProfitAndIdletime}により多項式時間で解くことができる.
ここでは,
{\graphUnit}で
全点の{\maxIdletime}が等しい場合の{\patProb}が(利得が異なっていても)多項式時間で解ける
ことを示す(定理\ref{theo:UnitUnaryIdletime}).

{\maxIdletime}が一般の場合については
多項式時間アルゴリズムやNP困難性を示すのが難しかったため,
2節で扱った{\timeSpecifiedPatProb}を再び考える.
地図が{\graphUnit}の場合は{\timeSpecifiedPatProb}がNP困難になることを示す
(定理\ref{theo:UnitExacIdletimeNPhard}).



\section{全点の{\maxIdletime}が等しい場合}

\begin{theo}
  \label{theo:UnitUnaryIdletime}
  地図が{\graphUnit}で全点の{\maxIdletime}が等しい場合,
  {\patProb}は(利得,巡査数が一般であっても)
  多項式時間で解くことができる.
\end{theo}

\begin{proof}
  {\graphUnit}は{\graphStar}の特殊な場合であるから,
  補題\ref{lemm:StarConditionOfGuarding}から
  {\graphUnit}の地図$(U, V)$の$V$の全点の{\maxIdletime}が$q$のとき,
  点集合$V$の任意の部分集合$W$について
  $W$を$m$人の巡査により警邏可能であることの必要十分条件は
  $d$を各辺の長さとして
  \[
    \sum_{v \in W} \min(d, q) = \card{W}\min(d, q) \leq mq
  \]
  である.

  地図が{\graphUnit}の場合,
  全点の{\maxIdletime}が等しいならば警邏する部分集合$W$は利得の大きい点から選べばよい
  (利得のより大きい点$v_1$とより小さい点$v_2$があるとき,
  $v_1$を警備して$v_2$を警備しない運行は常に$v_1$を警備する代わりに$v_2$を警備する運行に変換できる).
  $\card{W}\min(d, q) \leq mq$を満たす最大の$\card{W}$は
  $\card{W} = \left\lfloor {mq}/{\min(d, q)} \right\rfloor$であるので,
  利得の最も大きい$\lfloor {mq}/{\min(d, q)} \rfloor$点を選べばよい.
\end{proof}




\section{{\maxIdletime}が一般の場合}
\ref{chapter: star}節冒頭で述べた通り,
地図が{\graphStar}の場合については,
{\maxIdletime}が一般の場合は
{\patProb}は巡査が一人であってもNP困難であった\cite[Theorem~6]{coene2011charlemagne}.
このNP困難性の証明では
辺の長さが異なる{\graphStar}の地図を用いていた.
{\graphUnit}は{\graphStar}の辺の長さがすべて等しい場合であるため,
この方法によるNP困難性の証明ができない.
{\graphUnit}で{\maxIdletime}が一般の場合は
多項式時間アルゴリズムやNP困難性の証明が難しかったため,
{\graphLine}のときのように{\timeSpecifiedPatProb}を代わりに考える.




% 地図の形状が{\graphUnit}で巡査が一人で全点の利得が等しい場合,
% 全点の定時訪問が可能か判定する
% {\timeSpecifiedPatProbDecision}は多項式時間で解くことができる.

\begin{theo}
  \label{theo:UnitExacIdletimeNPhard}
  地図の形状が{\graphUnit}のとき,
  {\timeSpecifiedPatProb}は巡査が一人で全点の利得が等しい場合であってもNP困難である.
  \end{theo}
\begin{proof}
  % 最大独立集合問題において,
  % 無向グラフが与えられたときに独立点集合で最大のものを求めるが,
  % 間に辺の存在する2点の両方を選ぶことはできないという制約を,
  % {\patProb}において2点のどちらか一方しか警備できないという制約に変換する.
  NP困難であることが知られている最大独立集合問題からの帰着による.
  最大独立集合問題は,無向グラフが与えられたとき,
  どの二点間にも辺が存在しないような頂点集合(独立集合)のうち
  頂点の個数が最大のものを求める問題である.

  \newcommand{\primenum}[2]{p_{{#1}{#2}}}
  最大独立集合問題の入力として
  点集合$[n] = \{1, \ldots, n\}$,
  辺集合$E$のグラフ$G$が与えられたとする.
  同じ大きさの点集合$V$をもち,利得をすべて$1$,辺の長さをすべて$1$とした
  {\graphUnit}の地図$M = (U, V)$を考える.
  $M$の各点$i \in V$の{\exactTime}$(q_i, r_i)$を次のように定める.
  まず,$n(n - 1)/2$個の相異なる素数$\primenum{i}{j}\ (1 \leq i < j \leq n)$を用意する.
  $i > j$に対して$\primenum{i}{j}$と書くときは$\primenum{j}{i}$を指すことにする.
  各$i \in V$について,
  \begin{equation}
    q_i = \prod_{j \in [n] \setminus \{i\}} \primenum{i}{j}
  \end{equation}
  とし,
  $r _i$をすべての$j \in [n] \setminus \{i\}$に対して
  次を満たすように定める.
  \begin{equation}
    \label{equation: residues}
    r _i
    \equiv
    \begin{cases}
      1 & \text{$(i, j) \notin E$かつ$i > j$のとき} \\
      0 & \text{それ以外のとき}
    \end{cases}
    \pmod{p _{(i, j)}}
  \end{equation}
  そのような$r _i$は
  中国剰余定理より($q _i$の剰余として一意に)存在する.
  % 以上で得られた$M$と$(q _i, r _i) _{i \in [n]}$に対する
  % {\timeSpecifiedPatProb}の解は$G$の最大独立集合となる.

  $M$の異なる2点$i, j$の間の移動には時間$1$を要することから,
  その両方を定時訪問できるためには,
  訪問すべき時刻同士がすべて$1$以上離れていること,すなわち
  任意の整数$k, l$に対して$\abs{(k q_i + r_i) - (l q_j + r_j)} \geq 1$%
  が成り立つことが必要十分である.
  $q_i, r_i, q_j, r_j$がすべて整数のとき,これは
  $q_i k + r_i \neq q_j l + r_j$,
  すなわち
  $r_i - r_j \neq q_j l - q_i k$%
  が任意の整数$k, l$で成り立つことと同値である.
  $q _i$と$q _j$の最大公約数は$p _{(i, j)}$なので,
  これはさらに
  $r_i - r_j$が$p _{(i, j)}$の倍数でないこと,
  つまり$r_i \not\equiv r_j \mod p _{(i, j)}$に同値である.
  $r_i$の決め方\eqref{equation: residues}から,
  これは$(i, j) \notin E$に同値である.
  以上より,
  $(i, j) \in E$と$M$の2点$i, j$を両方定時訪問することができないこと
  が同値となるため,
  $M$の最大の警邏可能点集合は$G$の最大独立集合となることがわかる.

  また,
  $k$番目に小さい素数を$P_k$と書くと,$k \geq 6$について
  $P_k < k( \ln k + \ln\ln k )$が成り立つ\cite{dusart1999k}ので,
  % 或る数が素数であるかどうかを判定する多項式時間アルゴリズムが存在する\cite{agrawal2004primes}ので,
  $n(n - 1)/2$個の素数の列挙は$n$の多項式時間でできる.
\end{proof}



% 定期訪問

定理\ref{theo:UnitExacIdletimeNPhard}では,
各点の{\exactTime}$((q_1, r_1), \ldots, (q_n, r_n))$が与えられる場合についてNP困難性を示したが,
$(r_1, \ldots, r_n)$は与えられず$(q_1, \ldots, q_n)$のみが指定されている
次のような問題も考えることができる.

\begin{intervalSpecifiedPatrollingProblemDecision}
  巡査の人数$m$と地図$(U, V)$および
  各点の{\exactInterval}$(q_1, \ldots, q_n)$が与えられる.
  $m$人の巡査により$V$の全点を定期訪問可能か判定せよ.
  ただし,点$v$を\defword{\exactInterval}$q$で\defword{定期訪問}するとは,
  非負整数$r\ (0 \leq r < q)$が存在して
  $v$を{\exactTime}$(q, r)$で定時訪問することである.
\end{intervalSpecifiedPatrollingProblemDecision}

Campbellら\cite{campbell2005vehicle}は
地図が{\graphUnit}の場合,
{\intervalSpecifiedPatProbDecision}(\cite{campbell2005vehicle}ではDVMPDと称している)は
NP困難であることを示しており\cite[Theorem 4]{campbell2005vehicle},
さらに,河村らが
巡査が一人の場合に相当する問題がNP困難であることを示している\cite[Theorem 20]{kawamura2015simple}.

地図が{\graphUnit}で巡査が一人の場合,
{\timeSpecifiedPatProbDecision}は多項式時間で解けたが,
各点の訪問時刻を完全に指定するのではなく間隔のみを指定するように自由度を増した
{\intervalSpecifiedPatProbDecision}はNP困難になる.



% \begin{theo}
% \label{theo:NPhard_disjoint_residue_class_problem}
% グラフが{\graphUnit}のとき,
% {\intervalSpecifiedPatProbDecision}は巡査が一人であってもNP困難である.
% \end{theo}
% \begin{proof}
% Disjoint Residue Class Problem\cite{kawamura2015simple}からの帰着による.

% ある整数の組の集合$\{ (q_1, r_1), \ldots, (q_n, r_n) \}$が
% Disjoint Residue Class であるとは,
% 任意の整数$x$に対して$x \equiv r_i \mod q_i$となるような$i$が
% 高々1つ存在することと定義される.
% Disjoint Residue Class Problem とは
% 整数の組$(q_1, \ldots, q_n)$が与えられたときに,
% $\{ (q_1, r_1), \ldots, (q_n, r_n) \}$が
% Disjoint Residue Class となるような整数の組$(r_1, \ldots, r_n)$が
% 存在するかを判定する問題であり,
% NP困難であることが知られている\cite{kawamura2015simple}.

% 辺の長さがすべて$1$の{\graphUnit}のグラフ$M$を考えると,
% 巡査は各整数時刻にいずれか1点を訪問できる.
% $M$の各点$v_i$に{\exactTime}$(q_i, r_i)$が指定されているとき,
% 巡査一人で$M$の全点を警備可能であることの必要十分条件は,
% 訪問しなければならない時刻の集合族
% $\{ \{ q_i k + r_i \mid k \in \Zset \} \mid i \in \{ 1, \ldots, n \} \}$
% が互いに素であること,
% すなわち,整数の組の集合$\{ (q_1, r_1), \ldots, (q_n, r_n) \}$が
% Disjoint Residue Class であることである.

% よって,
% Disjoint Residue Class Problem の入力が$(q_1, \ldots, q_n)$のとき,
% 巡査の人数$1$と
% $n$点からなる{\graphUnit}のグラフ$M$で
% 各点$v_i$の{\exactInterval}を$q_i$,辺の長さがすべて$1$としたものを
% {\intervalSpecifiedPatProbDecision}の入力として与えると,
% これが警邏可能かどうかにより
% Disjoint Residue Class$\{ (q_1, r_1), \ldots, (q_n, r_n) \}$が
% 存在するかを判定できるので,Disjoint Residue Class Problemを帰着できる.
% \end{proof}
