\chapter{{\graphStar}}
\label{chapter: star}

地図が{\graphStar}の場合については,
利得か{\maxIdletime}のいずれかが一般であれば,
{\patProb}は巡査が一人であってもNP困難であることが
知られている\cite[Theorems 5 and 6]{coene2011charlemagne}.
よってここでは,
全点の利得と{\maxIdletime}が等しい場合について次のことを示す.

\begin{theo}
  \label{theo:StarUnaryProfitAndIdletime}
  地図が{\graphStar}で全点の利得と{\maxIdletime}が等しい場合,
  {\patProb}は(巡査の人数が一般であっても)
  多項式時間で解くことができる.
\end{theo}

{\independentPatProb}においては,地図が{\graphStar}で巡査の人数が一般の場合は
利得と{\maxIdletime}がすべて等しくてもNP困難になることが
Coeneらにより示されている\cite[Theorem~10]{coene2011charlemagne}.
{\graphLine}の場合では複雑な協力による警邏があり得たこと
(\ref{section:LineArbitraryIdletime}節)から考えると,
協力を許した方が簡単に解けるというのは意外な結果に思われる.
これは,
{\graphStar}の場合,
{\independentPatProb}では
うまく点集合を分割しなければならないことが難しさを生みNP困難になるのに対し,
{\patProb}では
後述の補題\ref{lemm:StarConditionOfGuarding}の証明中に述べる
単純な運行が可能となるためである.

図\ref{figure: graph_classes},\ref{figure: stars}で注意したように
{\graphStar}の中心は警邏すべき点ではないが,
本章では中心と点$v$を結ぶ辺(端点のうち中心は含まず$v$のみを含む)を$e_v$,
その長さを$d_v$と書く.
なお,中心も警邏すべき点である場合を考えるには,
$d_v = 0$であるような点$v$を追加すればよい.
また,$d_v = 0$のときは$e_v$は一点$v$とする.

\begin{lemm}
  \label{lemm:StarCostOfVertex}
  {\graphStar}において,
  {\maxIdletime}$q$の
  点$v$が警邏されているならば,任意の時刻$t \in \Rset$に対し,
  長さの和が$\min(2d_v, q)$であるような
  互いに交わらない有限個の時刻区間の合併$J \subseteq [t, t + q)$が存在し,
  $J$に属するすべての時刻において少なくとも一人の巡査が辺$e_v$または$v$上にいる.
\end{lemm}

\begin{proof}
  もし$2d_v > q$ならば,常に$e _v$上に巡査が存在する.
  何故なら,もし$e_v$上に巡査がいない時刻$\tau$があれば,
  長さ$2d_v$の時刻区間$(\tau - d _v, \tau + d _v)$にわたって$v$が放置され,
  仮定に反するからである.
  よって$J =[t, t + q)$とすればよい.
  以下では$2d_v \leq q$とする.

  警邏の条件から$v$は時刻区間$[t, t + q)$に少なくとも1回訪問される.
  もしその訪問時刻のうち$[t + d_v, t + q - d_v)$に属するもの$\tau$があれば,
  長さ$2 d _v$の時刻区間$J = (\tau - d _v, \tau + d _v)$にわたって
  巡査は辺$e_v$上におり,これは$[t, t + q)$に含まれる.

  そうでないとき,$v$は
  $[t, t + d_v)$か$[t + q - d_v, t + q)$に少なくとも1回訪問される.
  \begin{inparaenum}[(i)]
    \item $[t, t + d_v)$に訪問されるとき,
      $[t, t + d_v)$に属する最後の訪問時刻を$\tau$とすると,
      点$v$の警邏の条件と場合分けの条件から$\tau$の次の訪問時刻$\sigma$は
      $t + q - d_v < \sigma \leq \tau + q$を満たす.
      $\tau$と$\sigma$それぞれの前後$d_v$の時間のうち$[t, t + q)$に含まれる
      時刻区間$J = [t, \tau + d_v) \cup (\sigma - d_v, t + q)$%
      にわたって巡査は辺$e_v$に存在し,
      その長さは
      $q - \max(0, (\sigma - d_v) - (\tau + d_v))
        = \min(q, q - \sigma + \tau + 2d_v)
        \geq 2d_v$.
      \label{enum:firstInterval}
    \item $[t + q - d_v, t + q)$に1回以上訪問されるときも(\ref{enum:firstInterval})と同様.
    \qedhere
  \end{inparaenum}
\end{proof}


\begin{lemm}
  \label{lemm:StarConditionOfGuarding}
  地図が{\graphStar}で全点の{\maxIdletime}が$q$のとき,
  点集合$V$の部分集合$W$を警邏する$m$人の運行が存在するには,
  \begin{equation}
    \label{equation: star bound}
    \sum_{v \in W} \min(2d_v, q) \leq mq
  \end{equation}
  が成立つことが必要十分である.
\end{lemm}

\begin{proof}
  十分であることを示す.
  \eqref{equation: star bound}が成り立つとき,
  $m$人の巡査の運行$(a_1, \ldots, a_m)$を次のように定めれば$W$の全点が警邏される.
  $W' := \{ v \in W \mid 2d_v > q \}$,$l := m - \card{W \setminus W'}$とする.
  巡査$1$は中心から出発して速さ$1$で動きながら$W'$の全点をちょうど1度ずつ訪問し中心に戻るという巡回を繰り返す.
  中心点と点$v$の1回の往復には$2d_v$の時間を要するので,
  1回の巡回にかかる時間は
  $\sum_{v \in W'} 2d_v
    = \sum_{v \in W} \min(2d_v, q) - \card{W \setminus W'}q
    \leq (m - \card{W \setminus W'})q = lq$%
  より$lq$以下となる.
  巡査$i\ (i \in \{ 2, 3, \ldots, l \})$は巡査$1$より時間$(i - 1)q$遅れて
  同じ運行を行うようにする
  (すなわち,$a_i(t) = a_1(t - (i - 1)q)$となるように運行を定める).
  $a_1, \ldots, a_l$により$W'$の各点は時間$q$以上放置されることなく訪問される.
  $W \setminus W'$の各点は巡査$(l + 1), \ldots, m$が一人ずつ停止しこれを警邏する.
  これにより$W$の全点が警邏される.

  必要であることを示す.
  $W$が$m$人の巡査により警邏されているとすると,
  補題\ref{lemm:StarCostOfVertex}より,
  各点$v \in W$について,
  どの長さ$q$の時刻区間$[t, t + q)$にも合計$\min(2d_v, q)$の時間は
  いずれかの巡査が$e_v$上に存在し,
  その時間の合計は$\sum_{v \in W} \min(2d_v, q)$である.
  $m$人の巡査は各時刻$\tau \in [t, t + q)$に高々1つの辺に存在するので,
  全点警邏しているならば\eqref{equation: star bound}が成り立つ.
\end{proof}


補題\ref{lemm:StarConditionOfGuarding}より
{\graphStar}の任意の点集合$W$が警邏できるか否かを
$W$の点の隣接辺の長さだけから簡単に計算できることが分かった.
定理\ref{theo:StarUnaryProfitAndIdletime}では
全点の利得と{\maxIdletime}が等しい場合を考えているので,
警邏する部分集合としては隣接辺の短い点から順に選べばよい
(2点$v, w$について,$d_v > d_w$ならば,
$w$を警邏せず$v$を警邏する運行は$v$を警邏せず$w$を警邏する運行に必ず変換できる).
以上で定理\ref{theo:StarUnaryProfitAndIdletime}が示された.
