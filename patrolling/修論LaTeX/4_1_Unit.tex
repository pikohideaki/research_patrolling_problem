\chapter{{\graphUnit}}
\label{chapter: unit}

\ref{chapter: introduction}章で述べたように
{\graphUnit}は{\graphStar}の特殊な場合とみなせるため,
全点の利得と{\maxIdletime}が等しい場合,{\patProb}は
定理\ref{theo:StarUnaryProfitAndIdletime}により多項式時間で解くことができる.
ここでは,
{\graphUnit}で
全点の{\maxIdletime}が等しい場合の{\patProb}が(利得が異なっていても)多項式時間で解ける
ことを示す(定理\ref{theo:UnitUnaryIdletime}).

{\maxIdletime}が一般の場合については
多項式時間アルゴリズムやNP困難性を示すのが難しかったため,
\ref{chapter: line}章で扱った{\timeSpecifiedPatProb}を再び考える.
地図が{\graphUnit}の場合は{\timeSpecifiedPatProb}がNP困難になることを示す
(定理\ref{theo:UnitExacIdletimeNPhard}).



\section{全点の{\maxIdletime}が等しい場合}

\begin{theo}
  \label{theo:UnitUnaryIdletime}
  地図が{\graphUnit}で全点の{\maxIdletime}が等しい場合,
  {\patProb}は(利得,巡査数が一般であっても)
  多項式時間で解くことができる.
\end{theo}

\begin{proof}
  {\graphUnit}は{\graphStar}の特殊な場合であるから,
  補題\ref{lemm:StarConditionOfGuarding}から
  {\graphUnit}の地図$(U, V)$の$V$の全点の{\maxIdletime}が$q$のとき,
  点集合$V$の任意の部分集合$W$について
  $W$を$m$人の巡査により警邏可能であることの必要十分条件は
  $d$を各辺の長さとして
  \[
    \sum_{v \in W} \min(d, q) = \card{W}\min(d, q) \leq mq
  \]
  である.

  地図が{\graphUnit}の場合,
  全点の{\maxIdletime}が等しいならば警邏する部分集合$W$は利得の大きい点から選べばよい
  (2点$v, w$について,$w$より$v$の方が利得が大きい場合,
  $w$を警邏せず$v$を警邏する運行は$v$を警邏せず$w$を警邏する運行に必ず変換できる).
  $\card{W}\min(d, q) \leq mq$を満たす最大の$\card{W}$は
  $\card{W} = \left\lfloor {mq}/{\min(d, q)} \right\rfloor$であるので,
  利得の最も大きい$\lfloor {mq}/{\min(d, q)} \rfloor$点を選べばよい.
\end{proof}

{\graphStar}で全点の{\maxIdletime}が等しい場合は,
警邏可能な点の最大数が式\eqref{equation: star bound}で与えられることから,
利得が等しい場合は枝の短いものから選べばよく(定理\ref{theo:StarUnaryProfitAndIdletime}),
枝の長さが等しい場合は利得の大きいものから選べばよい(定理\ref{theo:UnitUnaryIdletime})
というようにまとめることができる.



