\chapter{{\graphUnit}}
\label{chapter: unit}

\ref{chapter: introduction}章で述べたように
{\graphUnit}は{\graphStar}の特殊な場合とみなせるため,
全点の利得と{\maxIdletime}が等しい場合,{\patProb}は
定理\ref{theo:StarUnaryProfitAndIdletime}により多項式時間で解くことができる.
ここでは,
{\graphUnit}で
全点の{\maxIdletime}が等しい場合の{\patProb}が(利得が異なっていても)多項式時間で解ける
ことを示す(定理\ref{theo:UnitUnaryIdletime}).

{\maxIdletime}が一般の場合については
多項式時間アルゴリズムやNP困難性を示すのが難しかったため,
\ref{chapter: line}章で扱った{\timeSpecifiedPatProb}を再び考える.
地図が{\graphUnit}の場合は{\timeSpecifiedPatProb}がNP困難になることを示す
(定理\ref{theo:UnitExacIdletimeNPhard}).



\section{全点の{\maxIdletime}が等しい場合}
\label{section: UnitUnaryIdletime}

\begin{theo}
  \label{theo:UnitUnaryIdletime}
  地図が{\graphUnit}で全点の{\maxIdletime}が等しい場合,
  {\patProb}は(利得,巡査数が一般であっても)
  多項式時間で解くことができる.
\end{theo}

\begin{proof}
  {\maxIdletime}を$q$とし,各辺の長さを$d$とする.
  {\graphUnit}は{\graphStar}の特殊な場合であるから,
  補題\ref{lemm:StarConditionOfGuarding}が適用できる.
  すなわち頂点集合$W$が$m$人で警邏できるためには,
  式\eqref{equation: star bound}が$d _v = d / 2$で成立つこと,
  つまり$W$に属する点の個数$\card{W}$が
  \begin{equation}
    \card{W} \cdot \min (d, q) \leq m q
  \end{equation}
  を満すことが必要十分である.
  したがって,利得の大きい順に
  $\lfloor m q / \min (d, q) \rfloor$個の点を選んだものが最適の$W$である.
\end{proof}

{\graphStar}で全点の{\maxIdletime}が等しい場合は,
警邏できる点の最大数が式\eqref{equation: star bound}で与えられることから,
利得が等しい場合は枝の短いものから選べばよく(定理\ref{theo:StarUnaryProfitAndIdletime}),
枝の長さが等しい場合は利得の大きいものから選べばよい(定理\ref{theo:UnitUnaryIdletime})
というようにまとめることができる.



