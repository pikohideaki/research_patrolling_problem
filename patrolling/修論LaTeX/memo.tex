[最終変更日時:
{\the\year/\the\month/\the\day, {\the\hour} 時{\the\minute}分}]

 [ToDo]
\begin{itemize}
  \item \textbf{木の場合など,今後の展望について}
  \item \textbf{図の追加・差し替え}
  \item \textbf{関連研究}
  \begin{itemize}
    \item 文献追加
    \item 目標が警邏・定時訪問・定期訪問などがあるが,"periodic ..."のような名前で警邏だったりすることもあるので注意を述べるべきか\kcomment{[はい、色々な呼称があることは列挙して述べると良いと思います。]}
    \item 警邏、配達、など色々な状況から考えられる問題であることを書いておく
  \end{itemize}
  \item 最後にチェック
  \begin{itemize}
    \item 参考文献リストの書式の統一
    \item 引用は著者名は2人以下なら全員列挙に
    \item 単語の統一
    \begin{itemize}
      \item 緑色にしている一時的な単語を決める
    \end{itemize}
    \item 補足(不要?)
    \begin{itemize}
      \item \ref{theo:LineUnaryIdletimePolyTimeSolvable}
        各区間の利得の計算や選ばれた区間に含まれる点の列挙の方法
      \item \ref{section:LineArbitraryIdletime}節:
        運行可能集合$S$に対して運行$a$が存在することの証明
    \end{itemize}
    \item 「章」、「節」(修正済み)
    \item 運行
  \end{itemize}
\end{itemize}
