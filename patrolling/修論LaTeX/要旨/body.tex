一人または複数の巡査が所与の領域を動き回り,その領域内の指定された場所を十分な頻度で訪問することで,これを守備,監督することを警邏という.

ここでは次の問題を考える.
辺に長さがついた無向グラフの上を幾人かの巡査が速さ$1$以下で動き回る.
各点には放置限度が定まっており,その時間以上放置してはならない(警邏の条件).
これが可能か判定せよ.


Coeneらは,
各点を警邏する巡査が高々一人という仮定の下で,
いくつかのグラフについて多項式時間アルゴリズムやNP困難性を示した.
本研究ではこの制約を無くした場合,
すなわち二人以上の巡査が協力して警邏する点があってもよい場合を考える.
一般のグラフではNP困難なので,
グラフの形状として線分,星,辺の長さがすべて等しい完全グラフを調べ,
これらはいずれも全点の放置限度が等しければ多項式時間で計算できるという結果を得た.
なお,星については
放置限度が一般の場合はNP困難になることが既に知られている.
効率的なアルゴリズムも見つからず困難性の証明も得られていない場合については,
各点の警邏の条件としている放置限度の代わりの制約も考えながら
計算量クラスの評価を試みる.
